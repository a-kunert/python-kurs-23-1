\documentclass[a4paper]{article} 
\usepackage{minted}
\usemintedstyle{friendly}
\addtolength{\hoffset}{-2.25cm}
\addtolength{\textwidth}{4.5cm}
\addtolength{\voffset}{-3.25cm}
\addtolength{\textheight}{5cm}
\setlength{\parskip}{0pt}
\setlength{\parindent}{0in}

%----------------------------------------------------------------------------------------
%	PACKAGES AND OTHER DOCUMENT CONFIGURATIONS
%----------------------------------------------------------------------------------------

\usepackage{blindtext} % Package to generate dummy text
\usepackage{charter} % Use the Charter font
%\usepackage{lmodern} % Use the Charter font
\usepackage[utf8]{inputenc} % Use UTF-8 encoding
\usepackage{microtype} % Slightly tweak font spacing for aesthetics
\usepackage[english, ngerman]{babel} % Language hyphenation and typographical rules
\usepackage{amsthm, amsmath, amssymb} % Mathematical typesetting
\usepackage{float} % Improved interface for floating objects
\usepackage[final, colorlinks = true, 
            linkcolor = black, 
            citecolor = black]{hyperref} % For hyperlinks in the PDF
\usepackage{graphicx, multicol} % Enhanced support for graphics
\usepackage{xcolor} % Driver-independent color extensions
\usepackage{marvosym, wasysym} % More symbols
\usepackage{rotating} % Rotation tools
\usepackage{censor} % Facilities for controlling restricted text
\usepackage{listings, style/lstlisting} % Environment for non-formatted code, !uses style file!
\usepackage{pseudocode} % Environment for specifying algorithms in a natural way
\usepackage{style/avm} % Environment for f-structures, !uses style file!
\usepackage{booktabs} % Enhances quality of tables
\usepackage{tikz-qtree} % Easy tree drawing tool
\usepackage{ifthen}
\usepackage{lastpage}
\usepackage{titlesec}
\usepackage{minted}
\usemintedstyle{friendly}
\tikzset{every tree node/.style={align=center,anchor=north},
         level distance=2cm} % Configuration for q-trees
\usepackage{style/btree} % Configuration for b-trees and b+-trees, !uses style file!
\usepackage[backend=biber,style=numeric,
            sorting=nyt]{biblatex} % Complete reimplementation of bibliographic facilities
\addbibresource{ecl.bib}
\usepackage{csquotes} % Context sensitive quotation facilities
\usepackage{fancyhdr} % Headers and footers
\pagestyle{fancy} % All pages have headers and footers
\fancyhead{}\renewcommand{\headrulewidth}{0pt} % Blank out the default header
\fancyfoot[L]{\thesheetsubmission{}} % Custom footer text
\fancyfoot[C]{} % Custom footer text
\fancyfoot[R]{\ifthenelse{\pageref{LastPage} > 1}{\footnotesize Seite \thepage{} von \pageref{LastPage} }} % Custom footer text
\newcommand{\note}[1]{\marginpar{\scriptsize \textcolor{red}{#1}}} % Enables comments in red on margin

\titleformat{\section}
{\normalfont\Large\bfseries}{Aufgabe~\thesection}{1em}{\normalsize}
\titlespacing*{\section}{0em}{6ex}{2ex}


\renewcommand{\theenumi}{\alph{enumi}}
\renewcommand\labelenumi{(\theenumi)}

\newcommand*{\sheetdate}[1]{\def\thesheetdate{#1}}
\newcommand*{\sheetsubmission}[1]{\def\thesheetsubmission{#1}}
\newcommand*{\sheetnumber}[1]{\def\thesheetnumber{#1}}
\newcommand{\py}[1]{\mintinline{python}{#1}}

%----------------------------------------------------------------------------------------


\sheetsubmission{}
\sheetnumber{2}
\sheetdate{11.11.2021}


\begin{document}

%-------------------------------
%	TITLE SECTION
%-------------------------------

\fancyhead[C]{}
\hrule \medskip % Upper rule
\begin{minipage}[t]{0.295\textwidth}
\raggedright
\footnotesize
Dr. Aaron Kunert \hfill\\   
aaron.kunert@salemkolleg.de \hfill \\
\end{minipage}
\begin{minipage}[t]{0.4\textwidth} 
\centering 
\large 
Einführung in Python\\ 
\normalsize 
Lösung zu Blatt \thesheetnumber{}\\ 
\end{minipage}
\begin{minipage}[t]{0.295\textwidth} 
\raggedleft
\footnotesize
\thesheetdate{}
\hfill\\
\end{minipage}
\medskip\hrule 
\bigskip

%-------------------------------
%	CONTENTS
%-------------------------------

\section{}
\begin{minted}{python}
day = 0
incidence = 219

above_300 = False
above_500 = False
above_1000 = False
above_10000 = False

# The loop simulates the incidence for every day
# Once it hits one of the given thresholds a message is printed
# To stop printing after the threshold is reached the flags above_xxx are used

while incidence <= 100000:
    day += 1
    incidence = int(incidence*1.0458)
    if incidence > 300 and not above_300:
        print(f"Über 300 an Tag {day}")
        above_300 = True
    if incidence > 500 and not above_500:
        print(f"Über 500 an Tag {day}")
        above_500 = True
    if incidence > 1000 and not above_1000:
        print(f"Über 1000 an Tag {day}")
        above_1000 = True
    if incidence > 10000 and not above_10000:
        print(f"Über 10000 an Tag {day}")
        above_10000 = True
    if incidence > 100000:
        print(f"Über 100000 an Tag {day}")

\end{minted}

\section{}
\begin{minted}{python}
example_list = [2, 7, 5, -1, 4, 12, 3, -19, 16]

# We add all items up, every second item is multiplied by 2 (i.e. weighting)
weighted_sum = 0
length = 0

for (index, number) in enumerate(example_list):
    if index % 2 == 0:
        # Remember: The first item in the list has index 0
        weighted_sum += number
        length += 1
    else:
        weighted_sum += 2*number
        length += 2

result = weighted_sum / length
print(f"Der gewichtete Durchschnitt ist {result}")
\end{minted}

\newpage
\section{} 
\begin{minted}{python}
my_name = "Aaron"
friends_name = "Tom"

example = ["Max", "Lara", "Kathrin", "Aaron", "Tom", "Sebastian"]     # should return True
example_2 = ["Max", "Lara", "Kathrin", "Tom", "Aaron", "Sebastian"]   # should return True
example_3 = ["Max", "Lara", "Kathrin", "Tom", "Sebastian", "Aaron"]   # should return False

my_list = example

length = len(my_list)

for (index, name) in enumerate(my_list):
    # if we are at the beginning of the list, there is nothing to check
    if index == 0:
        continue
    # Check if you are at the current position and your friend at the previous position
    if name == my_name and my_list[index - 1] == friends_name:
        print("Ja, die beiden Namen kommen hintereinander")
        break
    # Check if your friend is at the current position and you are at the previous position
    if name == friends_name and my_list[index - 1] == my_name:
        print("Ja, die beiden Namen kommen hintereinander")
        break
else:
    print("Nein, die beiden Namen kommen nicht hintereinander")
\end{minted}

\section{}
\begin{minted}{python}
counter = 1
n = 1000000000

while n != 1:
    counter += 1
    if n % 2 == 0:
        n = n//2
    else:
        n = 3*n + 1

print(f"Die Folge ist nach {counter} Folgengliedern zu Ende")
\end{minted}

\end{document}
