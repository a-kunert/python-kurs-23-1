\documentclass[a4paper]{article} 
\addtolength{\hoffset}{-2.25cm}
\addtolength{\textwidth}{4.5cm}
\addtolength{\voffset}{-3.25cm}
\addtolength{\textheight}{5cm}
\setlength{\parskip}{0pt}
\setlength{\parindent}{0in}

%----------------------------------------------------------------------------------------
%	PACKAGES AND OTHER DOCUMENT CONFIGURATIONS
%----------------------------------------------------------------------------------------

\usepackage{blindtext} % Package to generate dummy text
\usepackage{charter} % Use the Charter font
%\usepackage{lmodern} % Use the Charter font
\usepackage[utf8]{inputenc} % Use UTF-8 encoding
\usepackage{microtype} % Slightly tweak font spacing for aesthetics
\usepackage[english, ngerman]{babel} % Language hyphenation and typographical rules
\usepackage{amsthm, amsmath, amssymb} % Mathematical typesetting
\usepackage{float} % Improved interface for floating objects
\usepackage[final, colorlinks = true, 
            linkcolor = black, 
            citecolor = black]{hyperref} % For hyperlinks in the PDF
\usepackage{graphicx, multicol} % Enhanced support for graphics
\usepackage{xcolor} % Driver-independent color extensions
\usepackage{marvosym, wasysym} % More symbols
\usepackage{rotating} % Rotation tools
\usepackage{censor} % Facilities for controlling restricted text
\usepackage{listings, style/lstlisting} % Environment for non-formatted code, !uses style file!
%\usepackage{pseudocode} % Environment for specifying algorithms in a natural way
\usepackage{style/avm} % Environment for f-structures, !uses style file!
\usepackage{booktabs} % Enhances quality of tables
\usepackage{tikz-qtree} % Easy tree drawing tool
\usepackage{ifthen}
\usepackage{lastpage}
\usepackage{titlesec}
\tikzset{every tree node/.style={align=center,anchor=north},
         level distance=2cm} % Configuration for q-trees
\usepackage{style/btree} % Configuration for b-trees and b+-trees, !uses style file!
%\usepackage[backend=biber,style=numeric,
%            sorting=nyt]{biblatex} % Complete reimplementation of bibliographic facilities
%\addbibresource{ecl.bib}
\usepackage{csquotes} % Context sensitive quotation facilities
\usepackage{fancyhdr} % Headers and footers
\pagestyle{fancy} % All pages have headers and footers
\fancyhead{}\renewcommand{\headrulewidth}{0pt} % Blank out the default header
\fancyfoot[L]{\thesheetsubmission{}} % Custom footer text
\fancyfoot[C]{} % Custom footer text
\fancyfoot[R]{\ifthenelse{\pageref{LastPage} > 1}{\footnotesize Seite \thepage{} von \pageref{LastPage} }} % Custom footer text
\newcommand{\note}[1]{\marginpar{\scriptsize \textcolor{red}{#1}}} % Enables comments in red on margin

\titleformat{\section}
{\normalfont\Large\bfseries}{Lösung zu Aufgabe~\thesection}{1em}{\normalsize}
\titlespacing*{\section}{0em}{6ex}{2ex}


\renewcommand{\theenumi}{\alph{enumi}}
\renewcommand\labelenumi{(\theenumi)}

\newcommand*{\sheetdate}[1]{\def\thesheetdate{#1}}
\newcommand*{\sheetsubmission}[1]{\def\thesheetsubmission{#1}}
\newcommand*{\sheetnumber}[1]{\def\thesheetnumber{#1}}


%----------------------------------------------------------------------------------------


\sheetsubmission{Abgabe bis Mo. 14.11.2022 per Replit-Einladung}
\sheetnumber{2}
\sheetdate{02.11.2022}



\begin{document}

%-------------------------------
%	TITLE SECTION
%-------------------------------

\fancyhead[C]{}
\hrule \medskip % Upper rule
\begin{minipage}[t]{0.295\textwidth}
\raggedright
\footnotesize
Dr. Aaron Kunert \hfill\\   
aaron.kunert@salemkolleg.de \hfill \\
\end{minipage}
\begin{minipage}[t]{0.4\textwidth} 
\centering 
\large 
Einführung in Python\\ 
\normalsize 
Blatt \thesheetnumber{}\\ 
\end{minipage}
\begin{minipage}[t]{0.295\textwidth} 
\raggedleft
\footnotesize
\thesheetdate{}
\hfill\\
\end{minipage}
\medskip\hrule 
\bigskip

%-------------------------------
%	CONTENTS
%-------------------------------



\section{Durchschnitt berechnen}
Sei eine Liste von Zahlen gegeben. Berechne den gewichteten Durchschnitt, bei dem jede zweite Zahl in der Liste doppelt gewichtet werden soll.


\section{Liste auf Vorgänger/Nachfolger durchsuchen}
Sei ein Liste von Namen (z.B. \py{["Max", "Lara", "Tom"]}) sowie ein Name von einer deiner Freundinnen gegeben. Schreibe ein Programm, das prüft, ob die Namen von dir und deiner Freundin direkt hintereinander in der Liste vorkommen. 

\section{Sparraten}
Angenommen, Du hast ein Sparkonto, auf dem am 31.12.2022 ein Betrag von 3000€ liegt. Ab dem 1.1.2023 zahlst Du monatlich, jeweils zu Beginn des Monats, eine Sparrate von 50€ auf das Konto ein. Die Bank verspricht Dir \emph{monatlich} 0,28\% Zinsen auf Dein Guthaben, diese werden jeweils am Ende des Monats auf das Konto gutgeschrieben. Vom Konto wird nichts entnommen. 
 
\begin{enumerate}
\item Schreibe ein Skript, dass den Kontostand eines jeden Monats in den nächsten 15 Jahren ausgibt.
\item Ab welchem Zeitpunkt, ist der Zuwachs durch die Zinsen größer als durch die Sparrate?
\item In einem alternativen Modell, hast Du nur 10€ Sparrate, bekommst aber 0,9\% Zinsen im Monat. Nach welcher Laufzeit ist das alternative Modell profitabler?
\item Um welchen Betrag würden sich die beiden Modelle nach 20 Jahren unterscheiden?
\end{enumerate} 

\section{Das Collatz-Problem}
Sei $n$ eine beliebige, positive ganze Zahl. Folgende Vorschrift wird auf $n$ angewendet: Ist $n$ gerade, so ersetze man $n$ durch $\frac{n}{2}$, ist $n$ ungerade, so durch $3n + 1$. Setzt man diesen Prozess immer weiter fort, so erhält man eine Folge von positiven ganzen Zahlen. Sobald man bei der Zahl $1$ ankommt, wird dieser Prozess abgebrochen und die Folge gilt als beendet. Die sogenannte \emph{Collatz-Vermutung} besagt, dass diese Folge für jedes $n$ am Ende bei der Zahl $1$ ankommt. 

\vspace{2pt}

Zeige, dass die Vermutung für den speziellen Fall $n = 1.000.000.000$ korrekt ist. Wie lange ist diese Folge in diesem Fall, bis sie bei $1$ endet?


\vspace{2pt}

{\footnotesize\textbf{Beispiel:}
 Startet man bei $n = 20$ erhält man die Folge: $20, 10, 5, 16, 8, 4, 2, 1$. Die Folge hat hier die Länge $8$.}  

\vspace{2pt}

{\footnotesize\textbf{Bemerkung:}
Die Collatz-Vermutung ist bis heute ungelöst und inzwischen sind über 1.000.000 Euro Preisgeld für eine Lösung ausgehoben.}  
 


  








\end{document}
