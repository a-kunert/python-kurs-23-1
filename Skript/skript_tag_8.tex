\section{Funktionen \\ \footnotesize Wie man Code wiederverwerten kann}
\begin{frame}
\begin{block}{Problemstellung (Beispiel: Fieberthermometer)}
	\vspace{2pt}
	Gegeben seien Listen mit jeweils 3 Temperaturen (entspricht den Messwerten). 

	\pause

	Von jeder Liste soll zunächst der Durchschnitt gebildet werden. 
    Liegt dieser unter $37,8^\circ C$ so soll jeweils die Variable \py{result} auf \py{"Normale Temperatur"} im andern Fall auf \py{"Fieber"}
	gesetzt werden. 
		
	\pause
Die Listen sind 
	\begin{itemize}
		\item \py{temps_1 = [38, 37.8, 37.9]}
		\item \py{temps_2 = [37.5, 37.9, 38.1]}
		\item \py{temps_3 = [38.1, 37.7, 37.9]}
	\end{itemize}
\pause 
	Wie macht man das \emph{elegant}? 
\end{block}
\end{frame}
\begin{fragile}{}
\begin{block}{Lösung \onslide<8->{\footnotesize (Hauptsache es funktioniert)}}
\vspace{2pt}
\begin{overprint}
\onslide<2|handout:0>
\begin{minted}{python}
average = sum(temps_1) / len(temps_1)
















# 
\end{minted}
\onslide<3|handout:0>
\begin{minted}{python}
average = sum(temps_1) / len(temps_1)
if average >= 37.8: 
  result = "Fieber"
else: 
  result = "Normale Temperatur"












#  
\end{minted}
\onslide<4|handout:0>
\begin{minted}{python}
average = sum(temps_1) / len(temps_1)
if average >= 37.8: 
  result = "Fieber"
else: 
  result = "Normale Temperatur"

average_2 = sum(temps_2) / len(temps_2)










#
\end{minted}
\onslide<5|handout:0>
\begin{minted}{python}
average = sum(temps_1) / len(temps_1)
if average >= 37.8: 
  result = "Fieber"
else: 
  result = "Normale Temperatur"

average_2 = sum(temps_2) / len(temps_2)
if average_2 >= 37.8:
  result_2 = "Fieber"
else: 
  result_2 = "Normale Temperatur"






#
\end{minted}
\onslide<6|handout:0>
\begin{minted}{python}
average = sum(temps_1) / len(temps_1)
if average >= 37.8: 
  result = "Fieber"
else: 
  result = "Normale Temperatur"

average_2 = sum(temps_2) / len(temps_2)
if average_2 >= 37.8:
  result_2 = "Fieber"
else: 
  result_2 = "Normale Temperatur"

average_3 = sum(temps_3) / len(temps_3)




#
\end{minted}
\onslide<7-|handout:1>
\begin{minted}{python}
average = sum(temps_1) / len(temps_1)
if average >= 37.8: 
  result = "Fieber"
else: 
  result = "Normale Temperatur"

average_2 = sum(temps_2) / len(temps_2)
if average_2 >= 37.8:
  result_2 = "Fieber"
else: 
  result_2 = "Normale Temperatur"

average_3 = sum(temps_3) / len(temps_3)
if average_3 >= 37.8:
  result_3 = "Fieber"
else: 
  result_3 = "Normale Temperatur"
#
\end{minted}
\end{overprint}
\end{block}
\end{fragile}

\begin{frame}
\begin{block}{Nachteile dieser Lösung}
	\pause 
	\begin{itemize}[<+->]
	\item Viel Schreibarbeit, viel Wiederholung
	\item Der Code ist schwierig zu lesen. Man sieht vor lauter Wiederholungen nicht, was passiert. 
	\item Jedes Mal, wenn man diese \enquote{Berechnungslogik} verwendet, könnte man einen (Tipp-)Fehler machen.  
	\item Wenn man das Anforderungsprofil minimal ändert, muss diese \enquote{Logik} bei \emph{jedem} Auftreten im Code geändert werden
		(z.B. statt \py{"Fieber"} soll die Ausgabe \py{"Achtung Fieber!"}) heißen. In echten Projekten, kann das schnell ein paar Hundert Male sein. 	
	\end{itemize}
\end{block}
\end{frame}

\begin{fragile}
\begin{block}{Bessere Lösung}
\begin{minted}{python}
def check_temperature(temp_list):
  result = sum(temp_list) / len(temp_list)
  if result >= 37.8:
    result = "Fieber"
  else: 
    result = "Normale Temperatur"
  return result

result_1 = check_temperature(temps_1)
result_2 = check_temperature(temps_2)
result_3 = check_temperature(temps_3)
\end{minted}
\end{block}

\end{fragile}


\begin{frame}
\metroset{block=fill}
\begin{block}{Definition: Funktion}
\vspace{2pt}
Eine Funktion ist ein Codeblock, der nur ausgeführt wird, wenn die Funktion \emph{aufgerufen} wird. 
Man kann der Funktion Werte als \emph{Parameter} übergeben. 
Sie kann auch einen Wert als Ergebnis \emph{zurückgeben}. 
\end{block}

\pause 

\vspace{12pt}
Man kann sich eine Funktion wie eine Maschine vorstellen, wo man oben Dinge (=Parameter) hineinfüllt und unten ein Ergebnis (=Rückgabewert) herausbekommt. 
Unabhängig von dem Eingabe-Ausgabe-Prinzip, kann solch eine Maschine auch Nebeneffekte (z.B. Krach) produzieren. 

\pause 

\vspace{12pt}
Man unterscheidet zwischen \emph{Definition} und \emph{Ausführung} einer Funktion. 
\end{frame}



\begin{frame}
\metroset{block=fill}


\renewcommand{\baselinestretch}{1.5}
\begin{block}{Struktur der Funktions-Definition}	
\vspace{2pt}

\pause 

\texttt{def} \pause \textit{Funktionsname}\pause\texttt{(}\pause\textit{Parameter\_0}\pause, \textit{Parameter\_1}, \dots, \textit{Parameter\_n}\pause\texttt{)}\pause\texttt{:}\\
\pause \spacechar \spacechar \textit{Codezeile1} \\
\pause \spacechar \spacechar \textit{Codezeile2} \\
\pause \phantom{Code} \vdots \\
\pause \spacechar \spacechar \texttt{return} \textit{Ergebnis}
\end{block}
\renewcommand{\baselinestretch}{1}
\vspace{12pt}

\pause 

\begin{block}{Struktur eines Funktionsaufrufs}	
\vspace{2pt}
result = \textit{Funktionsname}\texttt{(}\textit{Argument\_0}, \textit{Argument\_1}, \dots, \textit{Argument\_n}\texttt{)}
\end{block}

\end{frame}


\begin{fragile}
\begin{exampleblock}{Beispiel}
\vspace{2pt}

\begin{overprint}
\onslide<1|handout:0>
\begin{minted}{python}
def square(x): 
  result = x * x
  return result

# Was passiert hier ? 




#
\end{minted}
\onslide<2|handout:0>
\begin{minted}{python}
def square(x): 
  result = x * x
  return result

# Was passiert hier ?
 
y = square(7)
print(y) 

#
\end{minted}
\onslide<3|handout:1>
\begin{minted}{python}
def square(x): 
  result = x * x
  return result

# Was passiert hier ?
 
y = square(7)
print(y) # 49

#
\end{minted}
\end{overprint}
\end{exampleblock}
\end{fragile}


\begin{frame}

\begin{block}{Good to know}
\pause 
\begin{itemize}[<+->]
	\item Eine Funktion muss schon \emph{vor} dem ersten Aufruf definiert worden sein (das ist nicht in allen Sprachen so). 
	\item Die Eingabwerte nennt man in der Funktionsdefintion \emph{Parameter}, beim Aufruf der Funktion nennt man sie jedoch \emph{Argumente}.
	\item Nicht jede Funktion braucht Eingangsdaten. Die Liste von Parametern einer Funktion kann daher leer sein.   
	\item Beim Aufruf spielt die Reihenfolge der angegebenen Argumente eine entscheidene Rolle. Sie werden entsprechend der Reihenfolge den Parametern in der Definition zugeordnet. 
%	\item Den Rückgabewert der Funktion erhält man durch den Zuweisungsoperator (\py{=}).
	\item Eine Funktion muss nicht unbedingt etwas zurückgeben, d.h. das \py{return}-Statement ist optional.
	\item Das \py{return}-Statement muss nicht unbedingt am Schluss der Funktion stehen. Jedoch wird Code, der nach dem \py{return}-Statement kommt, nicht mehr ausgeführt. 
\end{itemize}
\end{block}
\end{frame}

\begin{frame}{Übungen}

\begin{block}{Funktion mit einem Parameter}
\vspace{2pt}
Schreibe eine Funktion, die die übergebene Zahl verdoppelt und das Ergebnis zurückgibt. 
\end{block}
\vspace{12pt}
\begin{block}{Funktion mit zwei Parametern}
\vspace{2pt}
Schreibe eine Funktion, die die beiden übergebenen Zahlen multipliziert und das Ergebnis zurückgibt. 
\end{block}
\vspace{12pt}
\begin{block}{Funktion ohne Parameter}
	\vspace{2pt}
Schreibe eine Funktion, die Deinen Namen auf der Konsole ausgibt. 
\end{block}
\vspace{12pt}
\begin{block}{Funktion ohne Rückgabewert}
\vspace{2pt}
Was gibt eine Funktion zurück, die kein \py{return}-Statement enthält?
\end{block}
\end{frame}

\begin{frame}<beamer:0>[fragile]{Lösungen}


\begin{solutionblock}{Funktion mit einem Parameter}
\begin{minted}{python}
def double(number): 
  return number * 2
\end{minted}
\end{solutionblock}

\vspace{12pt}

\begin{solutionblock}{Funktion mit zwei Parametern}
\begin{minted}{python}
def multiply(number1,number2): 
  return number1 * number2
\end{minted}
\end{solutionblock}

\vspace{12pt}

\begin{solutionblock}{Funktion ohne Parameter}
\begin{minted}{python}
def my_name(): 
  print("Aaron Kunert")
\end{minted}
\end{solutionblock}

\end{frame}

\begin{frame}{Übung}
\begin{block}{Aggregatzustand von Wasser}
\vspace{2pt}
Schreibe eine Funktion, die die Temperatur als Paramter erwartet und abhängig von der Temperatur den Aggregatzustand von Wasser (\py{"fest"},\py{"flüssig"},\py{"gasförmig"}) als String zurückgibt. 
\pause 

\textbf{Zusatz:} Schaffst Du es, die Schlüsselwörter \py{elif} und \py{else} gar nicht und das Schlüsselwort \py{if} nur genau zweimal zu verwenden? 
\end{block}
\end{frame}

\begin{frame}<beamer:0>[fragile]{Lösung}

\begin{solutionblock}{Aggregatzustand von Wasser}
\begin{minted}{python}
def get_state(temp):
  if temp < 0:
    return "fest"
  if temp > 100:
    return "gasförmig"
  return "flüssig"
\end{minted}
\end{solutionblock}

\end{frame}


\begin{fragile}[Komplexere Übung]

\begin{block}{Gewichtete Durschnittsnote}
	\vspace{2pt}
Schreibe eine Funktion, die eine Liste der folgenden Struktur erwartet: 

\begin{minted}{python}
grades = [ 
  { 
    "subject": "Deutsch", 
    "grade": 14,
    "is_major": True
  },
  # ... 
  {
    "subject": "Sport",
    "grade": 11, 
    "is_major": False
  }
]
\end{minted}

Berechne die Durchschnittsnote. 

\textbf{Zusatz:} Berechne die Durchschnittsnote wenn Hauptfächer doppelt gewichtet werden. 
\end{block}

\end{fragile}


\begin{frame}<beamer:0>[fragile]{Lösung}

\begin{solutionblock}{Durchschnittsnote}
\begin{minted}{python}
def weighted_average(grades): 
  grades_sum = 0
  grades_length = 0
  for grade in grades: 
    grades_sum += grade["grade"]
    grades_length += 1
  result = grades_sum/grades_length
  return result 
\end{minted}
\end{solutionblock}

\end{frame}



\begin{frame}<beamer:0>[fragile]{Lösung}

\begin{solutionblock}{Zusatz: Gewichtete Durchschnittsnote}
\begin{minted}{python}
def weighted_average(grades): 
  grades_sum = 0
  grades_length = 0
  for grade in grades: 
    if grade["is_major"]: 
      grades_sum += 2 * grade["grade"]
      grades_length += 2
    else:
      grades_sum += grade["grade"]
      grades_length += 1
  result = grades_sum/grades_length
  return result 
\end{minted}
\end{solutionblock}

\end{frame}


\begin{fragile}[Übung]
\begin{block}{Zinsrechner}
\vspace{2pt}
Angenommen, Du hast 1000€ so angelegt, dass es darauf jeden Monat 0,5\% Zinsen gibt (Haha -- als ob!). 
Schreibe eine Funktion, die einen Geldbetrag erwartet und zurückgibt, nach wie vielen Monaten, dieser Geldbetrag erreicht wurde. 
\end{block}
\end{fragile}

\begin{frame}<beamer:0>[fragile]{Lösung}

\begin{solutionblock}{Zinsrechner}
\begin{minted}{python}
def months_until_rich(target):
    month = 0
    balance = 1000
    while balance < target:
        balance *= 1.005
        month += 1
    return month
\end{minted}
\end{solutionblock}
\end{frame}



\begin{fragile}[Übung]
\begin{block}{Zinsrechner mit Sparrate}
\vspace{2pt}
Angenommen, Du hast wieder 1000€ so angelegt, dass es darauf jeden Monat 0,5\% Zinsen gibt. Zusätzlich gibt es nun jedoch noch eine monatliche Sparrate von 25€, die ebenfalls auf das Konto eingezahlt wird. 
Schreibe eine Funktion, die einen Geldbetrag erwartet und zurückgibt, nach wie vielen Monaten, dieser Geldbetrag erreicht wurde. 
\end{block}
\end{fragile}

\begin{frame}<beamer:0>[fragile]{Lösung}

\begin{solutionblock}{Zinsrechner mit Sparrate}
\begin{minted}{python}
def months_until_rich(target):
    month = 0
    balance = 1000
    while balance < target:
        balance += 25
        balance *= 1.005
        month += 1
    return month
\end{minted}
\end{solutionblock}
\end{frame}



\begin{fragile}

\begin{block}{Optionale Parameter}
	
	\pause 
	
\vspace{2pt}
Manchmal wirst Du bei Funktionen bemerken, dass einige der Parameter fast immer den gleichen Wert haben. In diesem Fall, möchtest Du diese Parameter nicht bei jedem Aufruf immer hinschreiben, sondern nur dort, wo er vom Standardfall abweicht. Dies ist möglich, wenn man den Standardwert (\emph{default value}) bei der Definition mit angibt. 

\pause 

\textbf{Wichtig:} Bei der Definition müssen die optionalen Parameter immer hinter den Pflichtparametern stehen.
\end{block}

\vspace{12pt}
\pause 

\begin{exampleblock}{Beispiel}
\begin{minted}{python}
def double(number, factor=2): 
  return number * factor
\end{minted} 


\pause 

Diese Funktion ist sehr vielseitig: Im einfachen Fall verdoppelt sie die eingegebene Zahl. Optional lässt sich der Faktor aber beliebig verändern. 
\end{exampleblock}

\end{fragile}


\begin{frame}
\begin{block}{Typischer Einsatzbereich}
\vspace{2pt}
Oftmals merkt man im Verlauf eines Projektes, dass eine gegebene Funktion nicht flexibel genug ist, dann kann man sie um einen optionalen Parameter erweitern, ohne den bisherigen Code verändern zu müssen. 
\end{block}

\vspace{12pt}

\pause 

\begin{exampleblock}{Fiktives Beispiel}
\vspace{2pt}
Stell Dir vor, Du baust einen Zinsrechner wie oben, den auch schon einige Deiner Freund*innen, die den gleichen Sparplan haben, ebenfalls verwenden. Eine Deiner Freundinnen hat jedoch eine bessere Bank gefunden, die ihr 0,6\% Zinsen gibt. Also erweiterst Du die Funktion, so dass auch die Höhe der Zinsen anpassbar ist.   

\pause

Jedoch möchtest Du den bisherigen Code nicht verändern. Daher definierst Du den Zinssatz als optionalen Parameter, so dass die Funktion \enquote{abwärtskompatibel} zu ihrer bisherigen Verwendung ist.

\pause 
Die Definition startet dann mit \py{def months_until_rich(target, interest=0.5):} 
 
\end{exampleblock}

\end{frame}

\begin{frame}{Übung}
\begin{block}{Flexibler Zinsrechner}
\vspace{2pt}
Erweitere den Zinsrechner, so dass optional der Zinssatz angegeben werden kann. 	
\end{block}
\end{frame}

\begin{frame}<beamer:0>[fragile]{Lösung}

\begin{solutionblock}{Flexibler Zinsrechner}
\begin{minted}{python}
def months_until_rich(target, interest=0.5):
    month = 0
    balance = 1000
    while balance < target:
        balance += 25
        balance *= (1 + interest/100)
        month += 1
    return month
\end{minted}
\end{solutionblock}
\end{frame}




\begin{frame}{Schwierige Übung}

\begin{block}{Flexibler Durchschnittsrechner}
\vspace{2pt}
Erweitere die Funktion zur Berechnung von gewichteten Durchschnittsnoten so, dass optional der Gewichtungsfaktor angegeben werden kann. 	
\end{block}

\end{frame}

\begin{frame}<beamer:0>[fragile]{Lösung}

\begin{solutionblock}{Flexibler Durchschnittsrechner}
\begin{minted}{python}
def weighted_average(grades, weight=2): 
  grades_sum = 0
  grades_length = 0
  for grade in grades: 
    if grade["is_major"]: 
      grades_sum += weight * grade["grade"]
      grades_length += weight
    else:
      grades_sum += grade["grade"]
      grades_length += 1
  result = grades_sum/grades_length
  return result 
\end{minted}
\end{solutionblock}

\end{frame}

\begin{fragile}
	
\begin{block}{Named Parameters}
\vspace{2pt}
Hat eine Funktion viele Parameter, von denen etliche optional sind, so kann man einen Parameter statt über die Reihenfolge auch über den Namen übergeben. 
\end{block}

\pause 
\vspace{12pt}

\begin{exampleblock}{Beispiel}
\vspace{2pt}

\begin{minted}{python}
def my_function(parameter1, parameter2=0, parameter3="x", parameter4=-17):
  # ... 
\end{minted}
Möchte man jetzt die Funktion mit einem eigenen Wert \pybw{parameter1} und \pybw{parameter4} aufrufen aber alles andere auf Standard lassen, so geht das wie folgt: 

\py{my_function(15, parameter4=-20)}
\end{exampleblock}

	
\end{fragile}



\begin{frame}{Übung}
\begin{block}{Ganz flexibler Zinsrechner}
\vspace{2pt}
Erweitere den Zinsrechner, so dass zusätzlich optional der Startbetrag und die monatliche Sparrate angepasst werden können. 	

Welchen der Parameter muss man verdoppeln, um am schnellsten 10.000€ zu erreichen? 
\end{block}
\end{frame}

\begin{frame}<beamer:0>[fragile]{Lösung}

\begin{solutionblock}{Ganz flexibler Zinsrechner}
\begin{minted}{python}
def months_until_rich(target, interest=0.5, initial_amount=1000, savings_rate=25):
    month = 0
    balance = initial_amount
    while balance < target:
        balance += savings_rate
        balance *= (1 + interest/100)
        month += 1
    return month
\end{minted}
\end{solutionblock}
\end{frame}






























































