\section{Zu Beginn \dots}

\begin{frame}
\begin{block}{Kurze Vorstellungsrunde}
	\vspace{2pt}
	Schaffst Du es \emph{in 60 Sekunden} folgende Fragen möglichst knackig und aussagekräftig zu beantworten?
	\begin{itemize}
		\item Wer bist Du? 
		\item Windows, Mac oder Linux?
		\item Welche Vorkenntnisse hast Du beim Programmieren?  
		\item Warum hast Du Dich zum Python-Kurs angemeldet? 
		\item Wann wäre der Kurs für Dich perfekt gelaufen? (Best Case Szenario)
		\item Wann würdest Du den Kurs nicht weiter besuchen? (Worst Case Szenario)
	\end{itemize}
\end{block}
\end{frame}


\begin{frame}
	\begin{block}{Organisation des Kurses}
		\pause 
		\begin{itemize}[<+->]
			%\item Es soll 2 Nachholtermine geben. Zeiten dafür suchen wir im November. 
			\item Pausen: 30 Minuten für den Nachmittagssnack. Bei Bedarf mehr Pausen. Bitte danach pünktlich kommen! 
			\item Skript und alle Unterlagen findet man über MS Teams.
			\item Gelegentlich gibt es ein Aufgabenblatt (Teams) $\rightarrow$ Ca. 10 Tage Bearbeitungszeit, Abgabe durch Einladung zum Replit-Projekt  $\rightarrow$ individuelles Kurzfeedback
			\item Lernleistung durch: Anwesenheit, Mitarbeit und Bearbeitung der Aufgabenblätter 
			\item Wissenschaftliche Arbeit ist möglich
			\item Kommunikation per E-Mail
			\item Fragen sind immer und über alle Kanäle willkommen!
		\end{itemize}
	\end{block}
\end{frame}


\begin{frame}
	\begin{block}{Didaktik des Kurses}
		\begin{itemize}
			\item Mischung aus Vortrag, Präsenzübungen und Live-Coding
			\item Lösungen der Präsenzübungen gibt's im Handout in Teams
			\item Achtung: Präsenzübungen können erstmal frustrierend sein!
			\item Im Idealfall: Mehr Praxis statt Erklärungen
		\end{itemize}
	\end{block}
\end{frame}

\begin{frame}
	\begin{block}{Ziele des Kurses}
		\begin{itemize}
			\item Einblick in die \enquote{Denkweise} eines Computers
			\item Einige universelle Konzepte von Programmiersprachen kennenlernen
			\item Schulung des analytischen Denkens
			\item Verständnis von Python-Syntax
			\item Programmierung eines rudimentären Quizspiels
		\end{itemize}
	\end{block}
\end{frame}

\begin{frame}
	\begin{block}{Wo findet man Hilfe/Infos?}
		\vspace{2pt}
		\begin{itemize}
			\item Google
			\item \texttt{stackoverflow.com}
			\item Youtube (z.B. Tutorials)
			\item \texttt{docs.python.org/3}
			\item Bücher (z.B. \textit{Python Crashkurs} v. \textsc{Eric Matthes})
			\item \texttt{mailto: aaron.kunert@salemkolleg.de}
		\end{itemize}
	\end{block}
\end{frame}


\section{Was ist Python?}

\begin{frame}

\begin{block}{Wie funktioniert überhaupt die Programmierung in Python?}
	
	\begin{enumerate}
		\item Man \textbf{schreibt} eine Abfolge von Befehlen/Anweisungen in eine Text-Datei (nicht Word!) 
		\item Danach lässt man diese Datei vom Python-Interpreter \textbf{ausführen}. 
	\end{enumerate}
\end{block}
\end{frame}

\begin{frame}
\metroset{block=fill}
\uncover<+->{\begin{block}{Was wird benötigt?}
		\vspace{2pt}
		\uncover<+->{
			\textbf{Am Anfang}
			\begin{itemize}
				\item Compiler/Interpreter
				\item Texteditor (z.B. Mac: Xcode, Windows: Edit)
			\end{itemize}
		}
		\uncover<+->{
			\textbf{Später}
			\begin{itemize}
				\item Google
				\item Integrierte Entwicklungsumgebung (IDE)
				\item Versionskontrolle (VCS)
				\item Virtueller Maschinen
				\item Datenbanken
				\item Grafikbearbeitung
			\end{itemize}
		}
\end{block}}
\uncover<+->{
Editor und Compiler müssen nicht auf dem eigenem Computer installiert sein. Es gibt dafür auch cloudbasierte Lösungen. 
}
\end{frame}

\begin{frame}
	\begin{block}{Warum Python?}
		\begin{itemize}
			\item Einfaches Setup
			\item Einstiegsfreundliche Syntax
			\item Python ist eine Hochsprache
			\item Python muss nicht kompiliert, sondern nur interpretiert werden
			\item Große Community $\rightarrow$ großes \emph{Ecosystem}
			\item Python ist extrem vielseitig
			\item Python ist plattformunabhängig
		\end{itemize}
	\end{block}
\end{frame}

\begin{frame}
	\begin{block}{Typische Einsatzbereiche}
		\begin{itemize}
			\item Automatisierung
			\item Webscraping
			\item Datenanalyse
			\item Webentwicklung
		\end{itemize}
	\end{block}
\end{frame}





\section{Wie Programmierer denken \\ \footnotesize{Wie lernt man analytisches Denken?}}

\begin{frame}
\begin{quote}{Steve Jobs}
	Everyone in this country should learn to program a computer, because it teaches you to think.
\end{quote}
\end{frame}



\begin{frame}
\begin{block}{Phasen des Lernens einer Programmiersprache}
\vspace{2pt}
\uncover<+->{
	\begin{enumerate}[<+->]
		\item Annäherung:  Fokus auf dem Begreifen der Grundkonzepte
		\item Syntax: Fokus auf der korrekten Anwendung der Syntax
		\item Funktionalität: Fokus liegt darauf, Problemstellungen \emph{pragmatisch} zu lösen
		\item Design: Fokus auf les-und wartbaren Code
		\item Architektur: Fokus auf Strategie, Projekte nachhaltig und erweiterbar umzusetzen 
	\end{enumerate}
}
\end{block}
\end{frame}

\begin{frame}

\begin{block}{Problem Solving}
\vspace{2pt}
Sobald man die Syntax korrekt verwenden kann, steht das Lösen von Problemen beim Programmieren im Fokus.

\pause 

Dabei ist die Kunst nur wenige, klare begrenzte Bausteine (die Befehle der Sprache) \emph{kreativ} so zusammenzusetzen, damit das gegebene Problem gelöst wird. 
\end{block}
\end{frame}

\begin{frame}
\begin{block}{Problemlösungsstrategien}
	\pause 
	\begin{itemize}[<+->]
		\item \textbf{Trial} and Error
		\item Formuliere laut und möglichst präzise, was eigentlich die Problemstellung ist
		\item Zerlege das Problem in kleinere Probleme oder mach Dir Zwischenziele
		\item Gibt es schon eine ähnliches Problem, was Du gelöst hast und von wo aus Du starten kannst?
		\item Erkläre anderen das Problem und was Du schon bisher geschafft hast
		\item Ändere Dein Denken: Scheitern ist nicht das Ende des Weges, sondern der Anfang
		\item To be continued
	\end{itemize}
\end{block}
\end{frame}



\section{Die Konsole \\ \footnotesize{Das Sprachrohr zum Computer}}


\begin{frame}

\metroset{block=fill}
\begin{block}{Definition: Konsole}
	\vspace{2pt}
	Die Konsole ist ein simples Programm, das nur aus einem Eingabefeld besteht, und mit dem man mit einem anderen (in der Regel komplexeren Programm) mittels spezifischer Befehle kommunizieren kann. 
\end{block}

\vspace{10pt}

\pause 

\metroset{block=transparent}

\begin{exampleblock}{Beispiele}
	\begin{itemize}
		\item Windows-Eingabeaufforderung (Kommunikation mit Windows)
		\item Mac-Terminal (Kommunikation mit MacOs)
		\item Browser-Konsole (Kommunikation mit der Webseite mittels JavaScript)
		\item Die Python-Konsole
	\end{itemize}
\end{exampleblock}

\pause 

	Für Programmierer*innen ist die Konsole der wichtigste Kommunikationsweg zu ihrem Computerprogramm. 
\end{frame}


\begin{frame}{Übung}
\begin{block}{Überprüfe, ob Python bei Dir installiert ist}
	\begin{enumerate}
		\item Google wie man die Konsole bzw. das Terminal zum Betriebssystem öffnet 
		\item Öffne die Konsole
		\item Prüfe, ob Python installiert ist, indem Du einen der folgenden Befehle ausprobierst 
		\begin{itemize}
			\item \bash{python --version}
			\item \bash{python3 --version}
		\end{itemize}
		\item Interpretiere die Antwort
	\end{enumerate}	
\end{block}
\end{frame}




\section{Programmieren in der Cloud \\ \footnotesize{Schnell und unkompliziert einsteigen}}

\begin{frame}
\begin{block}{Browserbasierte IDE verwenden}
	\vspace{2pt}
\begin{enumerate}
\item Gehe auf https://replit.com
\item Erstelle ein Konto (Sign up)
\item Klicke auf \enquote{Create repl}
\item Wähle als Template \enquote{Python} aus
\end{enumerate}
\end{block}
\end{frame}


\section{Erste Schritte im REPL \\ \footnotesize (Read-Evaluate-Print-Loop)}


\begin{frame}
\begin{block}{Probier mal folgende Kommandos aus}	
	\begin{itemize}
		\item \py{3 + 4}
		\item \py{2 - 7}
		\item \py{"Hello" + "World"}
	\end{itemize}
\end{block}	
\end{frame}



\begin{frame}{Übung}
\uncover<+->{\begin{block}{Was machen die folgenden \textit{Operatoren}?}
	\begin{itemize}
		\item \pybw{+}
		\item \pybw{-}
		\item \pybw{*}
		\item \pybw{/}
		\item \pybw{**}
	\end{itemize}
\end{block}}
\uncover<+->{\begin{block}{Und diese?}
\begin{itemize}
		\item \%
		\item \pybw{//}
		\item \pybw{==}
		\item \pybw{<=}
		\item \pybw{<}
\end{itemize}
\end{block}}
\end{frame}

\begin{frame}<beamer:0>[fragile]
\frametitle{Lösungen}
\begin{solutionblock}{Operatoren I}
Die Operatoren 	\pybw{+} und \pybw{-} sind klar. Die Operatoren \pybw{*} und \pybw{/} bezeichnen Multiplikation und Division. 
Der Operator \pybw{**} berechnet die Potenz (hochnehmen). 
\end{solutionblock}

\vspace{12pt}

\begin{solutionblock}{Operatoren II}
	Der Operator \% ist der Modulo-Operator (vgl.  \href{https://de.wikipedia.org/wiki/Division_mit_Rest#Modulo}{Wikipedia}). Der Operator \pybw{//} arbeitet analog zur Divison, rundet das Ergebnis jedoch auf die nächste ganze Zahl ab. Die Operatoren \pybw{==} (Gleichheit), \pybw{<=} (Kleinergleich), \pybw{<} (kleiner als) sind Vergleichsoperatoren und geben entweder \pybw{True} oder \pybw{False} zurück
\end{solutionblock}


\end{frame}




\begin{frame}{Übung}
	\begin{block}{Wie rechnet Python?}
		\begin{itemize}
			\item Wird Punkt-vor-Strich berücksichtigt?
			\item Kann man mit Klammern die Reihenfolge beeinflussen?
			\item Was ist der Unterschied zwischen \py{10/5} und \py{10//5} ?
			\item Was bedeutet das Kommando \py{_}? 
			\item Wie kann man Zwischenergebnisse in Variablen speichern?
		\end{itemize}
	\end{block}
\end{frame}

\begin{frame}<beamer:0>[fragile]
\frametitle{Lösung}
\begin{solutionblock}{Rechenregeln}
	\begin{itemize}
		\item Python rechnet Punkt-vor-Strich.
		\item Python berücksichtigt Klammern.
		\item Das Ergebnis von \py{/} ist stets eine Fließkommazahl, das Ergebnis von \py{//} ist stets eine ganze Zahl. 
		\item Das Kommando \py{_}, referenziert das vorherige Ergebnis. 
		\item Zwischenergebnise lassen sich mittels des Zuweisungsoperators \py{=} in einer Variable speichern. 
	\end{itemize}
\end{solutionblock}
\end{frame}

\section{Variablen}

\begin{frame}
\uncover<+->{\begin{block}{}
		Jeder Wert in Python kann in einer Variable gespeichert werden: 
		
		\py{my_variable = 3}
\end{block}}

\uncover<+->{\begin{block}{}
		Die Zuweisung darf auch das Ergebnis einer Berechnung sein: 
		
		\py{my_new_variable = 3 + 5}
\end{block}}
\uncover<+->{\begin{block}{}
		Die Zuweisung darf auch weitere Variablen enthalten: 
		
		\py{my_brand_new_variable = my_variable + my_new_variable }
\end{block}}

\uncover<+->{\begin{block}{}
	Man darf auch Kettenzuweisungen machen: 
	
	\py{a = b = c = 100 }
\end{block}}
\end{frame}


\begin{frame}
\uncover<+->{\begin{block}{Gültige Variablennamen}
\begin{itemize}[<+->]
	\item Erlaubt sind Buchstaben (nur ASCII), Ziffern und Unterstriche
	\item Der Name darf nicht mit einer Ziffer starten
	\item Beliebige Länge 
	\item Wer's schon kennt als \emph{regulärer Ausdruck}:  \mintinline{php}{[_a-zA-Z][_0-9a-zA-Z]*}
	\item Schlüsselwörter sind nicht erlaubt
\end{itemize} 
\end{block}}
\vspace{12pt}
\uncover<+->{\begin{block}{Liste der Schlüsselwörter}
	\texttt{
	\begin{columns}[T,onlytextwidth]
		\column{0.2\textwidth}
		False\\ 	await\\ 	else\\ 	import\\ 	pass\\ assert \\	del\\ 	
		\column{0.2\textwidth}
		None \\	break \\	except \\ 	in \\	raise \\ global \\	not \\	 
		\column{0.2\textwidth}
		True \\	class \\ 	finally \\ 	is \\	return \\ with \\ async 
		\column{0.2\textwidth}
		and \\	continue \\ 	for \\	lambda \\	try \\ 	elif  \\	if  \\
		\column{0.2\textwidth}
		as \\ 	def \\ 	from  \\	nonlocal \\	while \\ 	or \\ 	yield
	\end{columns}
}
\end{block}}
\end{frame}


\begin{frame}
\uncover<+->{\begin{exampleblock}{Style-Guide Variablennamen}
	\begin{itemize}
		\item Englische Wörter
		\item Nur Kleinbuchstaben
		\item Möglichst aussdrucksstarke Namen verwenden
		\item Keine Angst vor langen Namen 
		\item Namen, die aus mehreren Worten bestehen, mit Unterstrich trennen (\textit{snake-case})
	\end{itemize}
	
	\uncover<+->{z.B. \py{students_in_this_room}, \py{number_of_unpaid_bills}}
\end{exampleblock}}

\end{frame}

\begin{frame}{Übung}
	
	\begin{block}{Probier's aus!}
		\begin{itemize}
			\item Welchen Wert hat eine Variable, wenn man sie nicht vorher definiert hat? 
			\item Was passiert, wenn man eine Variable definiert, die schonmal verwendet wurde?
			\item Wie kann man eine Variable mit Wert \py{3} um \py{1} vergrößern?
		\end{itemize}	
	\end{block}
\vspace{12pt}
\begin{solutionblock}{Lösung}
	\begin{itemize}
		\item Verwendet man eine undefinierte Variable, wird ein Fehler geworfen
		\item Ja, man kann eine Variable einfach neu definieren
		\item Hat beispielsweise \pybw{my_variable} den Wert 3, so lässt sich der Wert wie folgt vergrößern: 
		\pybw{my_variable = my_variable + 1} 
	\end{itemize}
\end{solutionblock}
\end{frame}
	

\section{Datentypen}

\begin{frame}
	\begin{block}{}
		Jeder Wert in Python hat einen \textit{Datentyp}. Unter anderem gibt es folgende \textit{primitive} Typen in Python.
		\begin{itemize}
			\item \py{int}  Integer (ganze Zahlen)
			\item \py{float} Float (Dezimalzahlen)
			\item \py{bool} Boolean (Wahrheitswerte)
			\item \py{str}  String (Zeichenketten)
			\item \pybw{NoneType} (Typ des leeren Werts \py{None})
		\end{itemize}
	\end{block}
\end{frame}


\begin{frame}
	\metroset{block=fill}
	
	\uncover<+->{\begin{block}{Integer}
		Ganze Zahlen wie z.B. \py{1}, \py{-1}, \py{0}. Nicht aber 
		\py{2.0} oder \py{0.0}. 	
	\end{block}}
	\vspace{12pt}
	\uncover<+->{\begin{block}{Float}
		Fließkommazahlen, z.B. \py{3.1415925}. Achtung: Bei Float-Berechnungen können schnell \enquote{Überraschungen} auftreten: Was ergibt z.B. \mintinline{python}{1.2 - 1.0} ? 
	\end{block}}
	\vspace{12pt}
	\uncover<+->{\begin{block}{Boolean}
		Booleans sind eine Sonderform von \py{int} und können nur die Werte \py{True} (entspricht 1) und \py{False} (entspricht 0) annehmen. Sie entstehen in der Regel, wenn man Fragen im Programm stellt (z.B. \py{3 < 4} oder \py{1 == 2}).   	
	\end{block}}
\end{frame}



\begin{frame}
	\metroset{block=fill}
	\uncover<+->{\begin{block}{String}
		Strings sind beliebige Zeichenketten und müssen in (ein-, zwei- oder dreifache) Anführungszeichen eingeschlossen werden. Die Ausdrücke \pybw{'hello'}, \pybw{"Hello"} und \pybw{"""Hello"""} sind (fast) äquivalent. 
	\end{block}}
	\vspace{12pt}
	\metroset{block=transparent}
	\uncover<+->{\begin{block}{Mehrzeilige Strings}
			\vspace{2pt}
		Ein \textit{Stringliteral} kann nur innerhalb einer Zeile definiert werden. Soll ein String mehrere Zeilen umfassen, müssen dreifache Anführungszeichen verwendet werden.  
	\end{block}}

	\end{frame}

	\begin{frame}
		\uncover<+->{\begin{block}{Steuerzeichen}
				\vspace{2pt}
			Gewisse Kombinationen mit Backslash sind reservierte Steuerzeichen. So bezeichnet beispielsweise \py{\n} einen Zeilenumbruch und \py{\t} ein Tabulatorzeichen. \\
			Beispiel: \py{"This text\nfills two lines"}
		\end{block}}
			\vspace{12pt}
		\uncover<+->{\begin{block}{Escaping}
			\vspace{2pt}
			Möchte man ein Steuerzeichen nicht ausführen, sondern buchstäblich nehmen. Muss man sie mit einem Backslash \textit{escapen} bzw. maskieren. \\
			Beispiel: \py{"This text fits in\\n one line"}
		\end{block}}
		\vspace{12pt}
		\uncover<+->{\begin{block}{Raw-Strings}
				\vspace{2pt}
				Möchte man alle Steuerzeichen eines Strings ignorieren, kann man ihn als \textit{Raw-String} definieren. \\
				Beispiel: \py{r"This \n String \t has no control characters"}
		\end{block}}
		
	\end{frame}

	
	\begin{frame}
		\metroset{block=fill}
		\uncover<+->{\begin{block}{Typecasting (Umwandlung von Typen)}
			\vspace{2pt}
			\uncover<+->{\textbf{Implizit}\\
			Bei manchen Operationen nimmt Python automatisch eine Typumwandlung vor. \\ Beispiel: \py{1 + 2.0} ergibt \py{3.0}	
		} \\ \\
		\uncover<+->{\textbf{Explizit \\}
			Die Funktionen \py{int()}, \py{float()}, \py{str()} und \py{bool()} führen jeweils eine Typumwandlung durch (sofern möglich). Beispiele: 
			\begin{itemize}
				\item \py{int(2.0)} ergibt \py{2} 
				\item \py{float(2)} ergibt \py{2.0} 
				\item \py{int("3")} ergibt \py{3}
			\end{itemize} 
		}
		\end{block}}
		
		\vspace{12pt}
		\metroset{block=transparent}
		\uncover<+->{\begin{block}{Typ einer Variablen ermitteln}
			\vspace{2pt}
			Mit der Funktion \py{type()} lässt sich der Typ bestimmen, z.B. \py{type(3.2)}.  	
		\end{block}}
		
	\end{frame}
	
	
	\begin{frame}{Übung}
		\begin{block}{Versuche die Fragen erst ohne Python zu beantworten, überprüfe Deine Vermutung}
			\begin{itemize}
				\item Welchen Datentyp hat das Ergebnis von \py{3 - 1.0} ?
				\item Was ist das Ergebnis von \py{"2" + 1} ? 
				\item Was ist das Ergebnis von \py{"2"} + \py{"2"}? 
				\item Sind die beiden Werte \py{0} und \py{"0"} gleich? 
				\item Sind die beiden Werte \py{2} und \py{True} gleich? 
				\item Sind die beiden Werte \py{bool(2)} und \py{True} gleich? 
				\item Sind die beiden Werte \py{1} und \py{True} gleich? 
			\end{itemize}
		\end{block}
	\end{frame}


\begin{frame}<beamer:0>[fragile]
\frametitle{Lösungen}
\begin{solutionblock}{Typaufgaben}
	\begin{itemize}
		\item Der Datentyp des Ergebnisses ist \pybw{float}
		\item Fehlermeldung
		\item Das Ergebnis ist \py{"22"}
		\item Nein
		\item Nein 
		\item Ja
		\item Ja
	\end{itemize}
\end{solutionblock}
\end{frame}

	\begin{frame}{Übung}
	
	\begin{block}{Erkläre mit Deinen eigenen Worten}
		\begin{itemize}
			\item Nach welcher Regel wandelt \py{int()} eine Fließkommazahl in eine ganze Zahl um? 
			\item Nach welchen Regeln wandelt \py{bool()} Zahlen und Strings in einen Wahrheitswert um? 
		\end{itemize}
	\end{block}
	
	
\end{frame}








