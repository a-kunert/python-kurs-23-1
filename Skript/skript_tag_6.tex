\begin{frame}
\begin{block}{Mutability}

\vspace{2pt}
Listen sind der erste Datentyp, den wir kennenlernen, der \emph{mutable} (veränderbar) ist. Die bisherigen Datentypen waren \emph{immutable}, d.h. man konnte sie zwar überschreiben, aber nicht verändern. 
\end{block}

\pause 
\vspace{12pt}

\metroset{block=fill}
\begin{block}{Call by Reference vs. Call by Value}
\vspace{2pt}
Enthält die Variable \py{my_list} eine Liste, so speichert Python eigentlich gar nicht die Liste in dieser Variable, sondern nur die Speicheradresse der Liste. 
Dieses vorgehen nennt man auch \emph{Call by Reference}. Bei den Datentypen \py{int} und \py{str} wird stattdessen tatsächlich der Wert der Variable abgespeichert. Dies nennt man \emph{Call by Value}.

\end{block}
\end{frame}

\begin{fragile}[Übung]
\begin{block}{Eine Liste kopieren}
\vspace{2pt}
Definiere die Variable \py{my_list} als die Liste \py{[1,2,3]}. Kopiere die Variable \py{my_list} in die Variable \py{my_list_copy}. Füge einen weiteren Eintrag zu \py{my_list} hinzu. Welchen Wert hat \py{my_list_copy}? 
\end{block}
\vspace{12pt}
\begin{solutionblock}{Lösung}
\begin{minted}{python}
my_list = [1, 2, 3]
my_list_copy = my_list
my_list.append(4)
print(my_list_copy) # 1 2 3 4 
\end{minted}
\end{solutionblock}
\end{fragile}



\begin{fragile}
\begin{block}{Schleife über Liste}
\vspace{2pt}
Analog wie über Strings und Ranges kann man Schleifen auch über eine Liste laufen lassen.  
\end{block}
\vspace{12pt}
\pause 

\begin{exampleblock}{Beispiel}
\vspace{2pt}
\begin{overprint}
\onslide<2|handout:0>
\begin{minted}{python}
countries = ["Bulgarien", "Griechenland", "Türkei", "Libanon"]

for country in countries:
  print(country)
\end{minted}
\onslide<3|handout:1>
\begin{minted}{python}
countries = ["Bulgarien", "Griechenland", "Türkei", "Libanon"]

for country in countries:
  print(country)
  
# Bulgarien
# Griechenland
# Türkei
# Libanon
\end{minted}
\end{overprint}
\end{exampleblock}
\end{fragile}


\begin{fragile}
\begin{block}{Schleife über Liste mit Indizes}
\vspace{2pt}
Möchte man in einer Schleife nicht nur die Listeneinträge, sondern auch die Indizes verwenden, so muss man die Funktion \py{enumerate()} auf die Liste anwenden. 
\end{block}
\vspace{12pt}
\pause 

\begin{exampleblock}{Beispiel}
\vspace{2pt}
\begin{overprint}
\onslide<2|handout:0>
\begin{minted}{python}
countries = ["Guatemala", "Nicaragua", "Honduras", "Belize"]

for (index, country) in enumerate(countries):
  print(f"Das {index + 1}. Land ist {country}")
\end{minted}
\onslide<3|handout:1>
\begin{minted}{python}
countries = ["Guatemala", "Nicaragua", "Honduras", "Belize"]

for (index, country) in enumerate(countries):
  print(f"Das {index + 1}. Land ist {country}")

# Das 1. Land ist Guatemala
# Das 2. Land ist Nicaragua
# Das 3. Land ist Honduras
# Das 4. Land ist Belize
\end{minted}
\end{overprint}
\end{exampleblock}
\end{fragile}


\begin{frame}{Übung}

\begin{block}{Liste durchsuchen}
	\vspace{2pt}
Prüfe, ob in einer Liste von Ländern das Land \py{"Italien"} vorkommt. Gib dazu auf der Konsole entweder 

\console{Italien ist in der Liste}

oder 

\console{Italien ist nicht in der Liste}

aus.  

\end{block}
\end{frame}


\begin{frame}<beamer:0>[fragile]{Lösung}
\begin{solutionblock}{Liste durchsuchen}
\begin{minted}{python}
# Wähle ein Beispiel für countries
countries = ["Finnland", "Norwegen", "Schweden", "Dänemark"]

for country in countries:
  if country == "Italien":
    print("Italien ist in der Liste")
    break
else: 
  print("Italien ist nicht in der Liste")
\end{minted}
\end{solutionblock}
\end{frame}

\begin{fragile}

\begin{block}{Ist ein Element in einer Liste enthalten?}
\vspace{2pt}	
Möchte man prüfen, ob ein Element in einer Liste enthalten ist, so kann man auch das Schlüsselwort \py{in} verwenden. 
\end{block}

\pause 
\vspace{12pt}

\begin{exampleblock}{Beispiel}
\vspace{2pt}
\begin{overprint}
\onslide<2|handout:0>
\begin{minted}{python}
countries = ["Finnland", "Norwegen", "Schweden", "Dänemark"]

var_1 = "Finnland" in countries
var_2 = "Deutschland" in countries

print(var_1)  
print(var_2)
\end{minted}
\onslide<3|handout:1>
\begin{minted}{python}
countries = ["Finnland", "Norwegen", "Schweden", "Dänemark"]

var_1 = "Finnland" in countries
var_2 = "Deutschland" in countries

print(var_1)  # True
print(var_2)  # False
\end{minted}
\end{overprint}

\end{exampleblock}
\end{fragile}


\begin{fragile}
	
\begin{block}{Eine Liste sortieren}
\vspace{2pt}
Um eine Liste zu sortieren, verwende die Methode \pybw{.sort()}. Dies verändert die Liste dauerhaft. \\
\pause 
Um eine sortierte Kopie einer Liste zu erstellen, verwende die Funktion \pybw{sorted()}.  \\
\pause 
Mit Hilfe des Parameters \pybw{reverse=True} lässt sich eine Liste absteigend ordnen. 
\end{block}	

\pause \vspace{12pt}

\begin{exampleblock}{Beispiel für \texttt{sort}}
\vspace{2pt}
\begin{overprint}
\onslide<4|handout:0>
\begin{minted}{python}
my_list = [1, 5, 2, 7]
my_list.sort()
print(my_list)  
\end{minted}
\onslide<5-|handout:1>
\begin{minted}{python}
my_list = [1, 5, 2, 7]
my_list.sort()
print(my_list)  # [1, 2, 5, 7]
\end{minted}
\end{overprint}

\end{exampleblock}

\vspace{12pt}

\pause \pause 

\begin{exampleblock}{Beispiel für \texttt{sorted}}
\vspace{2pt}
\begin{overprint}
\onslide<6|handout:0>
\begin{minted}{python}
my_list = [1, 5, 2, 7]
sorted_list = sorted(my_list)
print(my_list)  
print(sorted_list)  
\end{minted}
\onslide<7|handout:1>
\begin{minted}{python}
my_list = [1, 5, 2, 7]
sorted_list = sorted(my_list)
print(my_list)  # [1, 5, 2, 7]
print(sorted_list)  # [1, 2, 5, 7]
\end{minted}
\end{overprint}
\end{exampleblock}
\end{fragile}

\begin{fragile}
\begin{exampleblock}{Beispiel für absteigende Sortierung}
\vspace{2pt}
\begin{overprint}
\onslide<1|handout:0>
\begin{minted}{python}
my_list = [1, 5, 2, 7]
my_list.sort(reverse=True)
print(my_list)  

my_list = [7, 12, 5, 18]
sorted_list = sorted(my_list, reverse=True)
print(sorted_list) 
\end{minted}
\onslide<2|handout:1>
\begin{minted}{python}
my_list = [1, 5, 2, 7]
my_list.sort(reverse=True)
print(my_list)  # [7, 5, 2, 1]

my_list = [7, 12, 5, 18]
sorted_list = sorted(my_list, reverse=True)
print(sorted_list) # [18, 12, 7, 5] 
\end{minted}
\end{overprint}
\end{exampleblock}
\end{fragile}


\begin{fragile}[Übung]
	
\begin{block}{Beste/Schlechteste Note}
\vspace{2pt}
Sei \py{grades} eine Liste der Noten deiner letzten Klausuren (z.B. \py{grades = [12, 9, 14, 11]}). 
Gib dann auf der Konsole einmal die beste und einmal die schlechteste Note aus. 
\end{block}	

\vspace{12pt}

\begin{solutionblock}{Lösung}
\begin{minted}{python}
grades = [12, 9, 14, 11]
grades.sort()
min_grade = grades[0]
max_grade = grades[-1]
print(f"Schlechteste Note: {min_grade}")
print(f"Beste Note: {max_grade}")
\end{minted}
\end{solutionblock}
\end{fragile}

\begin{fragile}
\begin{block}{Nützliche Funktionen/Methoden}
\vspace{2pt}	
Für Listen stellt Python viele nützliche Methoden bzw. Funktionen bereit. Wenn Du googlest, findest Du für viele \enquote{Alltagsfragen} eine Lösung. 

Zum Beispiel hier: \texttt{https://docs.python.org/3/tutorial/datastructures.html}
\end{block}

\pause
\vspace{12pt}


\begin{exampleblock}{Beispiele}
\begin{minted}{python}
my_list = [2, 4, 8, 1]

len(my_list)   # = 4  (Gibt die Anzahl der Elemente an)
sum(my_list)   # = 15 (Berechnet die Summe der Elemente)
my_list.reverse() # [1, 8, 4, 2] (Dreht die Reihenfolge um)
my_list.insert(2,-1) # [2, 4, -1, 8, 1] (fügt den Wert -1 an Position 2 ein)
my_list.pop() # 1 (Gibt den letzten Eintrag der Liste zurück und entfernt ihn aus der Liste)
\end{minted}
\end{exampleblock}
\end{fragile}


\begin{fragile}[Übung]

\begin{block}{Durchschnittsnote}
\vspace{2pt}
Sei \py{grades} wieder eine Liste mit deinen letzten Noten.  Gib auf der Konsole die Durchschnittsnote aus. 
\end{block}	

\vspace{12pt}

\begin{solutionblock}{Lösung}
\begin{minted}{python}
grades = [12, 9, 14, 11]
total_sum = sum(grades)
count = len(grades)
average = total_sum/count
print(f"Die Durchschnittsnote ist {average}")
\end{minted}
\end{solutionblock}
	
\end{fragile}

\begin{frame}
\begin{block}{Slicing}
\vspace{2pt}
Wenn man eine Liste hat, ist es oft nötig, einen Teil der Liste \enquote{auszuschneiden}.\\
\pause
Dafür hat Python die \emph{Slice-Notation} eingeführt. \\
\pause 
Diese funktioniert nach folgendem Schema: 

\pause  \py{my_list[start:stop:step]}. 

\pause 
Die Einträge (start, stop, step) sind dabei jeweils optional. Wie immer wird der obere Wert (\pybw{stop}) gerade nicht erreicht.  
\pause 


Slicing lässt sich übrigens auch nach dem gleichen Schema auch auf Strings anwenden. 
\end{block}	

\pause
\vspace{12pt}

\begin{alertblock}{Wichtig}
\vspace{2pt}
Wenn man Slicing anwendet, erhält man eine Kopie der ausgewählten Elemente zurück. Die ursprüngliche Liste wird \emph{nicht} verändert. 
\end{alertblock}
\end{frame}


\begin{fragile}
\begin{exampleblock}{Beispiele}
	\vspace{2pt}
\begin{overprint}
\onslide<1|handout:0>
\begin{minted}{python}
my_list = [2, 4, 6, 8, 10]

my_list[1:3]    
my_list[0:4]     
my_list[1:1]     
my_list[0:4:2]   
my_list[:3]      
my_list[2:]      
my_list[:]      
my_list[1:-2]    
my_list[-3:-1]  
my_list[::-1]    
\end{minted}

\onslide<2|handout:0>
\begin{minted}{python}
my_list = [2, 4, 6, 8, 10]

my_list[1:3]     # [4, 6]
my_list[0:4]     
my_list[1:1]     
my_list[0:4:2]   
my_list[:3]      
my_list[2:]     
my_list[:]       
my_list[1:-2]   
my_list[-3:-1]  
my_list[::-1]    
\end{minted}

\onslide<3|handout:0>
\begin{minted}{python}
my_list = [2, 4, 6, 8, 10]

my_list[1:3]     # [4, 6]
my_list[0:4]     # [2, 4, 6, 8]
my_list[1:1]     
my_list[0:4:2]   
my_list[:3]      
my_list[2:]      
my_list[:]       
my_list[1:-2]    
my_list[-3:-1]  
my_list[::-1]    
\end{minted}

\onslide<4|handout:0>
\begin{minted}{python}
my_list = [2, 4, 6, 8, 10]

my_list[1:3]     # [4, 6]
my_list[0:4]     # [2, 4, 6, 8]
my_list[1:1]     # []
my_list[0:4:2]   
my_list[:3]      
my_list[2:]     
my_list[:]       
my_list[1:-2]    
my_list[-3:-1]   
my_list[::-1]    
\end{minted}


\onslide<5|handout:0>
\begin{minted}{python}
my_list = [2, 4, 6, 8, 10]

my_list[1:3]     # [4, 6]
my_list[0:4]     # [2, 4, 6, 8]
my_list[1:1]     # []
my_list[0:4:2]   
my_list[:3]      
my_list[2:]      
my_list[:]       
my_list[1:-2]    
my_list[-3:-1]   
my_list[::-1]    
\end{minted}

\onslide<6|handout:0>
\begin{minted}{python}
my_list = [2, 4, 6, 8, 10]

my_list[1:3]     # [4, 6]
my_list[0:4]     # [2, 4, 6, 8]
my_list[1:1]     # []
my_list[0:4:2]   # [2, 6]
my_list[:3]      
my_list[2:]      
my_list[:]       
my_list[1:-2]    
my_list[-3:-1]   
my_list[::-1]    
\end{minted}

\onslide<7|handout:0>
\begin{minted}{python}
my_list = [2, 4, 6, 8, 10]

my_list[1:3]     # [4, 6]
my_list[0:4]     # [2, 4, 6, 8]
my_list[1:1]     # []
my_list[0:4:2]   # [2, 6]
my_list[:3]      # [2, 4, 6]
my_list[2:]      
my_list[:]       
my_list[1:-2]   
my_list[-3:-1]   
my_list[::-1]    
\end{minted}

\onslide<8|handout:0>
\begin{minted}{python}
my_list = [2, 4, 6, 8, 10]

my_list[1:3]     # [4, 6]
my_list[0:4]     # [2, 4, 6, 8]
my_list[1:1]     # []
my_list[0:4:2]   # [2, 6]
my_list[:3]      # [2, 4, 6]
my_list[2:]      # [6, 8, 10]
my_list[:]       
my_list[1:-2]    
my_list[-3:-1]  
my_list[::-1]    
\end{minted}

\onslide<9|handout:0>
\begin{minted}{python}
my_list = [2, 4, 6, 8, 10]

my_list[1:3]     # [4, 6]
my_list[0:4]     # [2, 4, 6, 8]
my_list[1:1]     # []
my_list[0:4:2]   # [2, 6]
my_list[:3]      # [2, 4, 6]
my_list[2:]      # [6, 8, 10]
my_list[:]       # [2, 4, 6, 8, 10]
my_list[1:-2]    
my_list[-3:-1]   
my_list[::-1]    
\end{minted}

\onslide<10|handout:0>
\begin{minted}{python}
my_list = [2, 4, 6, 8, 10]

my_list[1:3]     # [4, 6]
my_list[0:4]     # [2, 4, 6, 8]
my_list[1:1]     # []
my_list[0:4:2]   # [2, 6]
my_list[:3]      # [2, 4, 6]
my_list[2:]      # [6, 8, 10]
my_list[:]       # [2, 4, 6, 8, 10]
my_list[1:-2]    # [4, 6]
my_list[-3:-1]    
my_list[::-1]     
\end{minted}

\onslide<11|handout:0>
\begin{minted}{python}
my_list = [2, 4, 6, 8, 10]

my_list[1:3]     # [4, 6]
my_list[0:4]     # [2, 4, 6, 8]
my_list[1:1]     # []
my_list[0:4:2]   # [2, 6]
my_list[:3]      # [2, 4, 6]
my_list[2:]      # [6, 8, 10]
my_list[:]       # [2, 4, 6, 8, 10]
my_list[1:-2]    # [4, 6]
my_list[-3:-1]   # [6, 8] 
my_list[::-1]     
\end{minted}

\onslide<12|handout:1>
\begin{minted}{python}
my_list = [2, 4, 6, 8, 10]

my_list[1:3]     # [4, 6]
my_list[0:4]     # [2, 4, 6, 8]
my_list[1:1]     # []
my_list[0:4:2]   # [2, 6]
my_list[:3]      # [2, 4, 6]
my_list[2:]      # [6, 8, 10]
my_list[:]       # [2, 4, 6, 8, 10]
my_list[1:-2]    # [4, 6]
my_list[-3:-1]   # [6, 8] 
my_list[::-1]    # [10, 8, 6, 4, 2]  
\end{minted}

\end{overprint}
\end{exampleblock}
\end{fragile}











