\section{Comprehensions}

\begin{frame}
\begin{block}{Typische Manipulationen}
\vspace{2pt}
Sehr häufig möchte man eine Datenstruktur (d.h. eine Liste oder ein Dictionary) basierend auf den Werten manipulieren. Dabei werden vor allem zwei Aspekte immer wieder gebraucht: Transformationen und Filter. 
\end{block}

\end{frame}

\begin{frame}
\metroset{block=fill}
\begin{block}{Transformation}
\vspace{2pt}
Ersetzt man jedes Element einer Liste durch ein aus dem ursprünglich berechnetem Element, so spricht man von einer \emph{Transformation} (engl. map/mapping). 
\end{block}

\pause 

\vspace{12pt}

\metroset{block=transparent}
\begin{exampleblock}{Beispiel}
\vspace{2pt}
Gegeben ist die Liste \py{my_list = [2, 5, 3, 12, 7] }. Die Liste soll so manipuliert werden, dass alle Einträge durch ihren doppelten Wert ersetzt werden. 
\end{exampleblock}

\end{frame}

\begin{fragile}
\begin{overprint}

\begin{block}{Traditionelle Lösung}
\vspace{2pt}
\begin{minted}{python}
my_list = [2, 5, 3, 12, 7]
result = []
for k in my_list: 
  result.append(2 * k)
print(result)
\end{minted}
\end{block}

\vspace{12pt}

\pause

\begin{block}{The Pythonian Way}
\vspace{2pt}
\begin{minted}{python}
my_list = [2, 5, 3, 12, 7]
result = [2 * k for k in my_list]
print(result)
\end{minted}
\end{block}
\end{overprint}
\end{fragile}



\begin{frame}
\metroset{block=fill}
\begin{block}{Filter}
\vspace{2pt}
Streicht man Elemente entsprechend ihres Wertes aus einer Liste, so spricht man von einem \emph{Filter}. 
\end{block}

\pause 

\vspace{12pt}

\metroset{block=transparent}
\begin{exampleblock}{Beispiel}
\vspace{2pt}
Gegeben ist die Liste \py{my_list = [2, 5, 3, 12, 7] }. Aus der Liste sollen alle ungeraden Einträge gestrichen werden. 
\end{exampleblock}

\end{frame}

\begin{fragile}
\begin{overprint}

\begin{block}{Traditionelle Lösung}
\vspace{2pt}
\begin{minted}{python}
my_list = [2, 5, 3, 12, 7]
result = []
for k in my_list: 
  if k % 2 == 0: 
    result.append(k)
print(result)
\end{minted}
\end{block}

\vspace{12pt}

\pause

\begin{block}{The Pythonian Way}
\vspace{2pt}
\begin{minted}{python}
my_list = [2, 5, 3, 12, 7]
result = [k for k in my_list if k % 2 == 0]
print(result)
\end{minted}
\end{block}
\end{overprint}


\end{fragile}

\begin{fragile}
\begin{block}{Kombination aus Transformation und Filter}
\vspace{2pt}
Selbstverständlich können Transformationen und Filter auch kombiniert werden. 
\end{block}

\vspace{12pt}

\begin{exampleblock}{Beispiel}
\vspace{2pt}
Lösche alle ungeraden Zahlen und verdopple dann alle Zahlen: 

\begin{minted}{python}
my_list = [2, 5, 3, 12, 7]
result = [2 * k for k in my_list if k % 2 == 0]
print(result)
\end{minted}
\end{exampleblock}

\end{fragile}


\begin{frame}
\metroset{block=fill}


\begin{block}{Struktur List Comprehension}	
\vspace{2pt}

\texttt{result = }
\pause {\Large\texttt{[}}
\pause\textit{Wunscheintrag(k)} \,
\pause \py{for} 
\pause \py{k} 
\pause \py{in} 
\pause \texttt{my\_list} 
\pause \py{if} 
\pause \textit{Bedingung(k)}
\pause \Large{\texttt{]}}

\end{block}

\end{frame}



\begin{fragile}

\begin{block}{Dictionary Comprehension}
\vspace{2pt}
Man kann das gleiche Verfahren auch auf Dictionaries anwenden. Dabei können jeweils key und value für die Transformationen und Filter verwendet werden. 
\end{block}

\vspace{12pt}

\begin{exampleblock}{Beispiel}
\vspace{2pt}
\begin{minted}{python}
my_dict = {"a": 2, "b": 3}
result = {key: value for (key, value) in my_dict.items()}
print(result)
\end{minted}
\end{exampleblock}

\end{fragile}

\begin{frame}
\metroset{block=fill}
\begin{block}{Struktur Dictionary Comprehension}	
\vspace{2pt}

\texttt{result = }
\pause {\Large\texttt{\{}}
\pause\textit{Wunsch-schlüssel(k,v)} 
\pause \,:\,
\pause  \textit{Wunsch-Wert(k,v)}  \\
\phantom{result = Klammer  }
\pause \py{for} 
\pause \py{(k,v)} 
\pause \py{in} 
\pause \py{my_dict.items()} 
\pause \py{if} 
\pause \textit{Bedingung(k,v)}
\pause \Large{\texttt{\}}}

\end{block}

\end{frame}



\begin{fragile}[Übung]

\begin{block}{List Comprehension}
\vspace{2pt}
Gegeben sei eine beliebige Liste von ganzen Zahlen. Streiche alle Zahlen, die ungerade oder negativ sind. Ersetze die übrigen Zahlen durch ihre Hälfte. 
\end{block}


\vspace{12pt}

\begin{solutionblock}{Lösung}
\begin{minted}{python}
my_list = [2, -3, 40, 15, 7, 8]
result = [k//2 for k in my_list if k % 2 == 0 and k >= 0]
\end{minted}
\end{solutionblock}

\end{fragile}


\begin{fragile}[Übung]

\begin{block}{Dictionary Comprehension}
\vspace{2pt}
Gegeben sei das Dictionary 
\py{{"Mathe": 9, "Sport": 13, "Physik": 4, "Bio": 12}}. 
Lösche nun daraus alle Noten unter 5 Punkte sowie die Sportnote. Zusätzlich soll das Dictionary danach wie folgt aussehen: 
\py{ {"In Mathe": "9 Punkte", "In Bio": "12 Punkte"} }. 

\end{block}


\vspace{12pt}

\begin{solutionblock}{Lösung}
\begin{minted}{python}
grades = {"Mathe": 9, "Sport": 13, "Physik": 4, "Bio": 12}
result = {
    "In " + key: f"{value} Punkte"
    for (key, value) in grades.items()
    if value >= 5 and key != "Sport"
}
\end{minted}
\end{solutionblock}

\end{fragile}










