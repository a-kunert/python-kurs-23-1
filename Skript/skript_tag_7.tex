
\begin{fragile}

\begin{block}{Schleife über Dictionary I}
\vspace{2pt}
Ähnlich wie bei Listen kann man Schleifen auch über ein Dictionary laufen lassen.  
\end{block}
\vspace{12pt}
\pause 

\begin{exampleblock}{Beispiel}
\vspace{2pt}
\begin{overprint}
\onslide<2|handout:0>
\begin{minted}{python}
capitals = {"Litauen": "Vilnius", "Lettland": "Riga", "Estland": "Tallin"}

for item in capitals:
  print(item)
\end{minted}
\onslide<3|handout:1>
\begin{minted}{python}
capitals = {"Litauen": "Vilnius", "Lettland": "Riga", "Estland": "Tallin"}

for item in capitals:
  print(item)

# Litauen
# Lettland
# Estland
\end{minted}
\end{overprint}
\end{exampleblock}
\end{fragile}

\begin{fragile}
\begin{block}{Schleife über Dictionary II}
\vspace{2pt}
Möchte man in der Schleife nicht nur die Schlüssel, sondern auch die Werte des Dictionaries zur Verfügung haben, so muss man die Methode \py{.items()} auf das Dictionary anwenden.   
\end{block}
\vspace{12pt}
\pause 


\begin{exampleblock}{Beispiel}
\vspace{2pt}
\begin{overprint}
\onslide<2|handout:0>
\begin{minted}{python}
capitals = {"Litauen": "Vilnius", "Lettland": "Riga", "Estland": "Tallin"}

for key, value in capitals.items():
  print(f"Hauptstadt von {key}: {value}")
\end{minted}
\onslide<3|handout:1>
\begin{minted}{python}
capitals = {"Litauen": "Vilnius", "Lettland": "Riga", "Estland": "Tallin"}

for key, value in capitals.items():
  print(f"Hauptstadt von {key}: {value}")

# Hauptstadt von Litauen: Vilnius
# Hauptstadt von Lettland: Riga
# Hauptstadt von Estland: Tallin
\end{minted}
\end{overprint}
\end{exampleblock}
\end{fragile}

\begin{frame}{Übungen}

\begin{block}{Zwei Dictionaries kombinieren}
	\vspace{2pt}
Gegeben seien zwei Dictionaries, z.B.  

\py{eu = {"Deutschland": "Berlin", "Frankreich": "Paris" }}

und 

\py{non_eu = {"Russland": "Moskau", "China": "Peking" }}

Füge die Einträge des zweiten Dictionaries zum ersten Dictionary hinzu. 
\end{block}

\pause 

\vspace{12pt}

\begin{block}{Ein Dictionary \enquote{filtern}}
\vspace{2pt}
Sei ein beliebiges Dictionary mit Noten gegeben. Entferne alle Einträge, deren Note schlechter als 5 Punkte ist. 
\end{block}
\end{frame}


\begin{frame}<beamer:0>[fragile]{Lösung}

\begin{solutionblock}{Zwei Dictionaries kombinieren}
\begin{minted}{python}
eu = {"Deutschland": "Berlin", "Frankreich": "Paris" }
non_eu = {"Russland": "Moskau", "China": "Peking" }

for key,value in non_eu.items():
  eu[key] = value
print(eu)
\end{minted}
\end{solutionblock}

\vspace{12pt}

\begin{solutionblock}{Ein Dictionary \enquote{filtern}}
\begin{minted}{python}
grades = {"Deutsch": 11, "Mathe": 3, "Sport": 14, "Geschichte": 1}
#  Man darf die Länge eines Dictionaries in einer Schleife nicht verändern, deshalb machen wir eine Kopie
result = dict(grades)
for key, value in grades.items():
  if value < 5:
    del result[key]
print(result)
\end{minted}
\end{solutionblock}

\end{frame}


\begin{fragile}

\begin{block}{Ein Dictionary zerlegen}
\vspace{2pt}
Mit der Methode \py{.keys()} erhält man eine Liste aller Schlüssel eines Dictionaries. \pause

Mit der Methode \py{.values()} erhält man eine Liste aller Werte eines Dictionaries. \pause 

In beiden Fällen, muss das Ergebnis mittels der Funktion \py{list()} in eine Liste umgewandelt werden. 
\end{block}


\vspace{12pt}
\pause 

\begin{exampleblock}{Beispiel}
\vspace{2pt}
\begin{overprint}
\onslide<4|handout:0>
\begin{minted}{python}
my_dictionary = {"China": "Peking", "Japan": "Tokio", "Korea": "Seoul"}

countries = my_dictionary.keys()
countries = list(countries)

capitals = my_dictionary.values()
capitals = list(capitals)

print(countries)
print(capitals)
\end{minted}
\onslide<5|handout:1>
\begin{minted}{python}
my_dictionary = {"China": "Peking", "Japan": "Tokio", "Korea": "Seoul"}

countries = my_dictionary.keys()
countries = list(countries)

capitals = my_dictionary.values()
capitals = list(capitals)

print(countries)  # ["China", "Japan", "Korea"]
print(capitals)   # ["Peking", "Tokio", "Seoul"]
\end{minted}
\end{overprint}
\end{exampleblock}
\end{fragile}













