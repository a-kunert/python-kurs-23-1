\section{Dictionaries}

\begin{frame}
\begin{block}{Problemstellung}
\vspace{2pt}
Eine Variable soll nicht nur die Namen von Ländern enthalten, sondern auch noch deren Hauptstadt. 

\vspace{8pt}

Wie macht man das? 
\end{block}
\end{frame}

\begin{fragile}{}
\begin{block}{Lösung}
	\begin{minted}{python}
	capitals = {"Deutschland": "Berlin", "Spanien": "Madrid", "Italien": "Rom"}
	
	country = input("Von welchem Land möchtest Du die Hauptstadt wissen?")
	
	print(f"Die Hauptstadt von { country } ist { capitals[country] } Punkte")
	\end{minted}
\end{block}
\end{fragile}

\begin{fragile}
	
	\metroset{block=fill}
	\begin{block}{Struktur eines \emph{Dictionaries}}
		\vspace{2pt}
		\large
		\texttt{my\_dict = }\pause {\Large\texttt{\{}}\pause 
		\texttt{key\_1}\pause\texttt{:}\pause\texttt{value\_1}\pause,
		\pause 
		\texttt{key\_2:value\_2}, \pause 
		\dots   
		, \texttt{key\_n:value\_n}\pause \Large{\texttt{\}}}
	\end{block}
	\pause 
	
	Das \emph{Dictionary} \py{my_dict} enthält Schlüssel-Wert-Paare (\emph{key-value-pairs}). Die Schlüssel müssen eindeutig und unveränderlich sein (z.B. vom Typ \pybw{string} oder \pybw{int}). Die Werte dürfen beliebige Datentypen sein. 
	
\end{fragile}

\begin{fragile}

\begin{block}{Good to know}
	\pause
\begin{itemize}[<+->]
\item Zur besseren Übersichtlichkeit werden Dictionaries oftmals wie folgt formatiert: 
\begin{minted}{python}
capitals = { 
  "Deutschland": "Berlin", 
  "Spanien": "Madrid", 
  "Italien": "Rom"
}
\end{minted}
\item Dictionaries sind mutable, können also verändert werden. 
\item Dictionaries besitzen keine vernünftige Anordnung und können nicht geordnet werden. 
\item Ein Dictionary kann leer sein. 
\end{itemize}
\end{block}
\end{fragile}

\begin{fragile}
\begin{itemize}
\item Oftmals bietet es sich an, statt einem Dictionary eine Liste von Dictionaries zu verwenden:
\begin{minted}{python}
countries = [
  { 
    "name": "Deutschland", 
    "capital": "Berlin",
    "pop": 82000000,
    "is_eu_member": True
  },
  # ... 
  {
    "name": "Italien",
    "capital": "Rom", 
    "pop": 65000000,
    "is_eu_member": True
  }
]
\end{minted}
\end{itemize}
\end{fragile}



\begin{frame}


\begin{block}{Auf Dictionary-Elemente zugreifen}
	
	\vspace{2pt}
	
	Sei \py{my_dict = {"a": 5, "b": 8}}.
	
	\pause
	
	Mit der Syntax \py{my_dict["a"]} kann man den Wert an der Stelle \py{"a"} auslesen. 
	
	\pause 
	
	Mit der Syntax \py{my_dict["a"] = 12} kann  man einzelne Werte des Dictionaries verändern. 
	
	\pause 
	
	Auf diese Weise können auch ganz neue Paare hinzugefügt werden. Zum Beispiel: \py{my_dict["c"] = -2}. 
\end{block}
\end{frame}


\begin{frame}{Übung}

\begin{block}{Dictionary manipulieren}
	\vspace{2pt}
	Gegeben sei das folgende Dictionary: 
	
	\py{grades = {"Mathe": 8, "Bio": 11, "Sport": 13}} 
	
	Bestimme die Durchschnittsnote dieser drei Fächer. Verbessere danach Deine Mathenote um einen Punkt und füge noch eine weitere Note für Englisch hinzu (Abfrage über Konsole). Gib danach erneut den Durchschnitt an.  
\end{block}

\end{frame}


\begin{frame}<beamer:0>[fragile]{Lösung}

\begin{solutionblock}{Dictionary manipulieren}
\begin{minted}{python}
grades = {"Mathe": 8, "Bio": 11, "Sport": 13}

grades_sum = grades["Mathe"] + grades["Bio"] + grades["Sport"]
average = grades_sum/len(grades)
print(f"Der Durchschnitt ist {average} Punkte")

grades["Mathe"] += 1

eng_grade = input("Welche Note hast Du in Englisch? ")
eng_grade = int(eng_grade)
grades["Englisch"] = eng_grade

grades_sum += grades["Englisch"]
average = grades_sum/len(grades)
print(f"Der Durchschnitt ist {average} Punkte")
\end{minted}
\end{solutionblock}
\end{frame}



\begin{frame}
\begin{block}{Einen Eintrag aus einem Dictionary entfernen}
\vspace{2pt}
Wie bei Listen, kann man mittels \py{del}-Statement einen Eintrag aus einem Dictionary entfernen: 

\py{del eu_countries["united_kingdom"]}	

\end{block}	



\end{frame}	

\begin{fragile}
\begin{block}{Was wird hier passieren?}
\vspace{2pt}
\begin{minted}{python}
old_capitals = {"Deutschland": "Bonn", "Norwegen": "Oslo"}
new_capitals = old_capitals

new_capitals["Deutschland"] = "Berlin" 

print(old_capitals)
print(new_capitals)
\end{minted}
\end{block}

\pause 
\vspace{12pt}

\begin{block}{Erklärung}
	\vspace{2pt}
Da Dictionaries mutable sind, findet bei ihnen der Aufruf mittels \emph{Call by Reference} statt. Das heißt, dass in der Variable \py{old_capitals} bzw. \py{new_capitals} nicht die Länder gespeichert sind, sondern nur die Speicheradresse, wo die Länder zu finden sind. Ändert man die zugrundeliegenden Daten an einer Stelle, so ändern sie sich daher auch an der anderen Stelle. 
\end{block}


\end{fragile}

\begin{frame}
	\begin{block}{Eine Kopie von einem Dictionary erstellen}
		\vspace{2pt}
		Mit der Funktion \py{dict()} kann man eine Kopie von einem Dictionary erstellen. 
		
		Beispiel: \py{dict(my_dict)} erstellt eine Kopie von \py{my_dict}.
	\end{block}
\end{frame}

\begin{fragile}
\begin{block}{Schleife über Dictionary I}
\vspace{2pt}
Ähnlich wie bei Listen kann man Schleifen auch über ein Dictionary laufen lassen.  
\end{block}
\vspace{12pt}
\pause 

\begin{exampleblock}{Beispiel}
\vspace{2pt}
\begin{overprint}
\onslide<2|handout:0>
\begin{minted}{python}
capitals = {"Litauen": "Vilnius", "Lettland": "Riga", "Estland": "Tallin"}

for item in capitals:
  print(item)
\end{minted}
\onslide<3|handout:1>
\begin{minted}{python}
capitals = {"Litauen": "Vilnius", "Lettland": "Riga", "Estland": "Tallin"}

for item in capitals:
  print(item)

# Litauen
# Lettland
# Estland
\end{minted}
\end{overprint}
\end{exampleblock}
\end{fragile}

\begin{fragile}
\begin{block}{Schleife über Dictionary II}
\vspace{2pt}
Möchte man in der Schleife nicht nur die Schlüssel, sondern auch die Werte des Dictionaries zur Verfügung haben, so muss man die Methode \py{.items()} auf das Dictionary anwenden.   
\end{block}
\vspace{12pt}
\pause 


\begin{exampleblock}{Beispiel}
\vspace{2pt}
\begin{overprint}
\onslide<2|handout:0>
\begin{minted}{python}
capitals = {"Litauen": "Vilnius", "Lettland": "Riga", "Estland": "Tallin"}

for key, value in capitals.items():
  print(f"Hauptstadt von {key}: {value}")
\end{minted}
\onslide<3|handout:1>
\begin{minted}{python}
capitals = {"Litauen": "Vilnius", "Lettland": "Riga", "Estland": "Tallin"}

for key, value in capitals.items():
  print(f"Hauptstadt von {key}: {value}")

# Hauptstadt von Litauen: Vilnius
# Hauptstadt von Lettland: Riga
# Hauptstadt von Estland: Tallin
\end{minted}
\end{overprint}
\end{exampleblock}
\end{fragile}

\begin{frame}{Übungen}

\begin{block}{Zwei Dictionaries kombinieren}
	\vspace{2pt}
Gegeben seien zwei Dictionaries, z.B.  

\py{eu = {"Deutschland": "Berlin", "Frankreich": "Paris" }}

und 

\py{non_eu = {"Russland": "Moskau", "China": "Peking" }}

Füge die Einträge des zweiten Dictionaries zum ersten Dictionary hinzu. 
\end{block}

\pause 

\vspace{12pt}

\begin{block}{Ein Dictionary \enquote{filtern}}
\vspace{2pt}
Sei ein beliebiges Dictionary mit Noten gegeben. Entferne alle Einträge, deren Note schlechter als 5 Punkte ist. 
\end{block}
\end{frame}


\begin{frame}<beamer:0>[fragile]{Lösung}

\begin{solutionblock}{Zwei Dictionaries kombinieren}
\begin{minted}{python}
eu = {"Deutschland": "Berlin", "Frankreich": "Paris" }
non_eu = {"Russland": "Moskau", "China": "Peking" }

for key,value in non_eu.items():
  eu[key] = value
print(eu)
\end{minted}
\end{solutionblock}

\vspace{12pt}

\begin{solutionblock}{Ein Dictionary \enquote{filtern}}
\begin{minted}{python}
grades = {"Deutsch": 11, "Mathe": 3, "Sport": 14, "Geschichte": 1}
#  Man darf die Länge eines Dictionaries in einer Schleife nicht verändern, deshalb machen wir eine Kopie
result = dict(grades)
for key, value in grades.items():
  if value < 5:
    del result[key]
print(result)
\end{minted}
\end{solutionblock}

\end{frame}


\begin{fragile}

\begin{block}{Ein Dictionary zerlegen}
\vspace{2pt}
Mit der Methode \py{.keys()} erhält man eine Liste aller Schlüssel eines Dictionaries. \pause

Mit der Methode \py{.values()} erhält man eine Liste aller Werte eines Dictionaries. \pause 

In beiden Fällen, muss das Ergebnis mittels der Funktion \py{list()} in eine Liste umgewandelt werden. 
\end{block}


\vspace{12pt}
\pause 

\begin{exampleblock}{Beispiel}
\vspace{2pt}
\begin{overprint}
\onslide<4|handout:0>
\begin{minted}{python}
my_dictionary = {"China": "Peking", "Japan": "Tokio", "Korea": "Seoul"}

countries = my_dictionary.keys()
countries = list(countries)

capitals = my_dictionary.values()
capitals = list(capitals)

print(countries)
print(capitals)
\end{minted}
\onslide<5|handout:1>
\begin{minted}{python}
my_dictionary = {"China": "Peking", "Japan": "Tokio", "Korea": "Seoul"}

countries = my_dictionary.keys()
countries = list(countries)

capitals = my_dictionary.values()
capitals = list(capitals)

print(countries)  # ["China", "Japan", "Korea"]
print(capitals)   # ["Peking", "Tokio", "Seoul"]
\end{minted}
\end{overprint}
\end{exampleblock}
\end{fragile}













