\section{Die For-Schleife \\ \footnotesize Einen Programmabschnitt x-mal ausführen}




\begin{frame}

\begin{block}{Problemstellung}
\vspace{2pt}
Lies eine ganze Zahl \py{x} ein. Gib dann folgende Zeilen auf der Konsole aus 

\texttt{1}\\
\texttt{2}\\
\texttt{3}\\
\texttt{4}\\
\vdots \\
\texttt{x}

\vspace{12pt}
Wie macht man das? 

\end{block}
\end{frame}

\begin{fragile}{}
	\begin{block}{Lösung \footnotesize (fast)}
		\begin{minted}{python}
			x = input("Gib eine Zahl ein")
			x = int(x)
			
			for k in range(1, x):	
			  print(k)
		\end{minted}
	\end{block}
\end{fragile}

\begin{frame}

	\renewcommand{\baselinestretch}{1.5}
	\metroset{block=fill}
	\begin{block}{Struktur der \texttt{for...in} Schleife}
		\vspace{2pt}
		\pause \py{for} \pause \textit{Variable} \pause \py{in} \pause \py{range}(\textit{start}, \textit{end})\pause\texttt{:} \pause \\
		\spacechar\spacechar Codezeile 1 \pause \\ 
\spacechar\spacechar Codezeile 2 \pause \\
\spacechar\spacechar \phantom{Code} \vdots \pause  \\
\textit{Code, der nicht mehr Teil der Schleife ist}
	\end{block}

\vspace{12pt}
\pause 

\metroset{block=transparent}
	\renewcommand{\baselinestretch}{1}
	\begin{block}{Wie funktioniert's?}
		\vspace{2pt}
	Die Schleifenvariable wird zunächst gleich dem unteren Wert in \py{range} gesetzt. Dann wird der \pybw{for}-Block wiederholt ausgeführt. Bei jedem Durchgang wird die Schleifenvariable um \pybw{1} vergrößert und zwar so lange, wie der Wert der Schleifenvariable kleiner als der obere Wert in \py{range} ist. 	
	\end{block}
\end{frame}

\begin{fragile}
\begin{exampleblock}{Beispiel}
\vspace{2pt}
	
	\begin{overprint}
		\onslide<1|handout:0>
\begin{minted}{python}
for x in range(1,5)
  print(2*x)
\end{minted}
\onslide<2|handout:1>
\begin{minted}{python}
for x in range(1,5)
  print(2*x)

# prints 
# 2
# 4
# 6
# 8
\end{minted}
\end{overprint}
\end{exampleblock}
\end{fragile}


\begin{frame}
\begin{block}{Good to know}
	\pause
	\begin{itemize}[<+->]
		\item Achtung: Die Schleifenvariable erreicht nie das obere Ende der \py{range}-Funktion, sondern bleibt immer \pybw{1} drunter. 
		\item Die \py{range}-Funktion ist nicht auf 1er-Schrittweite beschränkt. Mit folgendem Ausdruck werden die Zahlen von \py{0} bis \py{9} z.B. in 3er-Schritten durchlaufen: \py{range(0, 10, 3)}. 
		\item \texttt{For}-Schleifen sind flexibel und können alles mögliche durchlaufen, z.B. auch die einzelnen Buchstaben eines Strings (dazu später mehr).
	\end{itemize}
\end{block}
\end{frame}


\begin{fragile}[Übung zum Einstieg]
	
\begin{block}{Eingangsbeispiel}
\vspace{2pt}
Schreibe ein Skript, das alle Zahlen von 1 bis 100 auf der Konsole ausgibt. 

\vspace{12pt}
\begin{solutionblock}{Lösung}
\begin{minted}{python}
for k in range(1, 101)
  print(k)
\end{minted}
\end{solutionblock}
\end{block}
\end{fragile}



\begin{fragile}[Übungen]

\begin{block}{Zählen}
\vspace{2pt}
Zähle auf der Konsole in 7-er Schritten bis 70.

\console{7} \\
\console{14} \\
\phantom{|} \vdots\\
\console{70}
\end{block}


\pause 

\vspace{12pt}

\begin{block}{Einmaleins: Die 7er-Reihe}
	\vspace{2pt}
	Schreibe ein kleines Skript, was die 7er-Reihe (bis 70) wie folgt auf der Konsole ausgibt: 
	
	\console{1 mal 7 ist 7}\\	
	\console{2 mal 7 ist 14}\\
	\phantom{4 mal} \vdots  
\end{block}
\end{fragile}


\begin{frame}<beamer:0>[fragile]{Lösungen}


\begin{solutionblock}{Zählen}
\begin{minted}{python}
for k in range(1, 11):
  print(7*k)
\end{minted}
\end{solutionblock}

\vspace{12pt}

\begin{solutionblock}{Einmaleins}
\begin{minted}{python}
for k in range(1, 11):
  print(f"{k} mal 7 ist {7 * k}")
\end{minted}
\end{solutionblock}

\end{frame}

\begin{fragile}[Komplexe Übungen]

\begin{block}{Zählen in krummen Abständen}	
\vspace{2pt}
Zähle auf der Konsole bis 20, allerdings sollen nur Zahlen ausgegeben werden, die durch 3 oder durch 5 teilbar sind:

\console{3} \\
\console{5} \\
\console{6} \\
\console{9} \\
\console{10}\\
\phantom{|}\vdots\\
\console{20}
\end{block}

\vspace{12pt}
\pause

\begin{block}{Anzahl bestimmen}
\vspace{2pt}
Bestimme die Anzahl der Zahlen zwischen 1 und 20, die durch 3 oder durch 5 teilbar sind. 
\end{block}
\end{fragile}

\begin{frame}<beamer:0>[fragile]{Lösungen}


\begin{solutionblock}{Zählen in krummen Abständen}
\begin{minted}{python}
for k in range(1, 21):
  if k % 3 == 0 or k % 5 == 0: 
    print(k)
\end{minted}
\end{solutionblock}

\vspace{12pt}

\begin{solutionblock}{Anzahl bestimmen}
\begin{minted}{python}
counter = 0
for k in range(1, 21):
  if k % 3 == 0 or k % 5 == 0:
    counter = counter + 1 
print(f"Es gibt {counter} gesuchte Zahlen")
\end{minted}
\end{solutionblock}

\end{frame}


\begin{fragile}[Schwierigere Übungen]
	
\begin{block}{Das Gauss-Problem}
\vspace{2pt}	
Berechne die Summe der Zahlen 1 bis 100. 
\end{block}
\vspace{12pt}
\begin{solutionblock}{Lösung}
\begin{minted}{python}
result = 0
for k in range(1, 101):
  result = result + k
print(f"Das Ergebnis ist {result}.")
\end{minted}
\end{solutionblock}

	
	
\end{fragile}


\begin{fragile}[Übung]
\begin{block}{Schleife über einen String}
\vspace{2pt}
Lies Deinen Namen auf der Konsole ein und gib die Buchstaben einzeln auf der Konsole auf. 
\end{block}
\vspace{12pt}
\begin{solutionblock}{Lösung}
\begin{minted}{python}
name = input("Gib Deinen Namen ein: ")

for letter in name:
  print(letter)
\end{minted}
\end{solutionblock}
\end{fragile}




\begin{fragile}[Übung]
\begin{block}{Needle-Haystack-Problem}
\vspace{2pt}
Lies Deinen Namen auf der Konsole ein und überprüfe, ob er den Buchstaben \emph{a} (groß/klein) enthält. 
\end{block}
\vspace{12pt}
\begin{solutionblock}{Lösung}
\begin{minted}{python}
name = input("Gib ein Wort ein: ")

# Flag (Schalter) initialisieren
name_contains_letter_a = False

for letter in name:
  if letter == "a" or letter == "A":
    name_contains_letter_a = True

if name_contains_letter_a:
  print("Der Name enthält ein 'a'.")
else:
  print("Der Name enthält kein 'a'.")
\end{minted}
\end{solutionblock}
\end{fragile}



\begin{fragile}[Übung mit Trick]
\begin{block}{Quersumme}
	\vspace{2pt}
	Lies eine ganze Zahl \py{x} ein und bestimme ihre Quersumme. 
	
	\textbf{Tipp:} Wandle die Zahl zunächst in einen String um \\
\end{block}

\vspace{12pt}

\begin{solutionblock}{Lösung}
\begin{minted}{python}
number = input("Gib eine Zahl ein: ")
result = 0
# Wir lassen die Zahl als String, damit wir eine Schleife über die Ziffern legen können
for digit in number:
  result = result + int(digit)  # Achtung: digit ist ja eigentlich ein String

print(f"Die Quersumme von {number} ist {result}")
\end{minted}
\end{solutionblock}

\end{fragile}


\begin{fragile}[Brutale Übung]
	
	
\begin{block}{Fibonacci-Zahlen}
\vspace{2pt}
Die Zahlenfolge $1,1,2,3,5,8,13\ldots$ nennt man \emph{Fibonacci}-Folge. Dabei ensteht ein Element der Folge, durch die Addition des vorherigen und vorvorherigen Elements. 

\vspace{1pt}

Berechne die 30. Fibonacci-Zahl.  
\end{block}
\vspace{12pt}
\begin{solutionblock}{Lösung}
\begin{minted}{python}
last = 1  # letzte Zahl
current = 1  # aktuelle Zahl

for k in range(2, 31):
  old_current = current  # Zahl zwischenspeichern
  current = current + last
  last = old_current

print(f"Die {k}-te Fibonacci-Zahl ist {current}")
\end{minted}
\end{solutionblock}
	
\end{fragile}










	




